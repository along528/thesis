\chapter{Conclusions}
\label{sec:conclusion}

\section{Summary}
The first study searching for the $WWW$ production process 
was presented using the 2012 dataset from the LHC,
with an emphasis on the details of the fully-leptonic
decay channel. The search was able to achieve good signal
and background discrimination given the very small initial
signal to background ratio. The results were interpreted as a
measurement on the SM cross-section, though the precision
of the measurement is limited by statistics. A study of the
semi-leptonic channel was also performed, though it
is not the focus of this thesis. It has a similar sensitivity
to the fully-leptonic channel and so slightly improves the precision
on the cross-section measurement when combining the two results. The 
combined observed cross-section is measured to be
$227.03 ~ ^{+202}_{-198} \stat ~^{+154}_{-160} \syst ~\textrm{fb}$,
which is consistent with the expected SM value of 241.47 fb.
Though with statistical and systematic uncertainties around
90\% and 70\%, respectively, it is also largely consistent
with a SM signal cross-section of zero. Thus, it is still too early
to claim evidence of the SM signal.

Still, we can say with confidence that there is not a strong excess of the
SM signal observed in the data. This is made explicit by looking 
at predictions of new physics manifested as aQGCs
in an EFT framework. Limits at 95\% CL were set on two non-unitarized aQGC parameters,
which were observed 
to be 
$-1.27 \times 10^3 \TeV^{-4} < f_{S,0}/\Lambda^4 < 1.76 \times 10^3 \TeV^{-4}$
and
$-2.10 \times 10^3 \TeV^{-4} < f_{S,1}/\Lambda^4 < 2.71 \times 10^3 \TeV^{-4}$
for the combined channels when assessing the limits independently.
These are the first limits for these parameters performed in the 
$WWW$ production channel. Additional limits were presented
considering the impact of unitarization using a form factor as a function
of the unitarization scale.
In the most extreme case, when $\Lambda = 500 \GeV$, the limits
worsen by a factor of 5 to 8. 


\section{Outlook}
The LHC is currently operating in Run 2 with a higher center of mass energy
of $\sqrt{s} = 13 \TeV$. The cross-sections for the $WWW$ process
are expected to improve by roughly a factor of two when moving from 8 TeV
to 13 TeV, though this is also true for most of 
the backgrounds to the process.  It is also planned to have significantly
more data by the end of Run 2, with possibly as much as 300 fb$^{-1}$ by 
the end of 2018.  This could result in an increase in the amount
of signal and background with respect to the 2012 analysis
by a factor of around 30, which would be equivalent to more than 100 signal
events if the 
same signal selection is used. As a result, if the signal to background ratio
stays the same, it should 
be possible to claim evidence of the $WWW$ process by the end of Run 2 
if it does indeed exist. 

