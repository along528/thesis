\begin{abstractpage}
In 2012 a resonance with a mass of 125 GeV resembling the elusive Higgs boson was discovered simultaneously by the ATLAS and CMS experiments using data collected from the Large Hadron Collider (LHC) at CERN.  Its observation finally confirms the mechanism for Spontaneous Electroweak Symmetry Breaking (EWSB) necessary for describing the mass structure  of the electroweak (EW) gauge bosons.  In 2013, Peter Higgs and Francois Englert were awarded the Nobel Prize in physics for their work in developing this theory of EWSB now referred to as the Higgs mechanism.  The explanation for EWSB is often referred to as the last piece of the puzzle required to build a consistent theory of particle physics known as the Standard Model.  But does that mean that there are no new surprises to be found?  Many EW processes have yet to be measured and are just starting to become accessible with the data collected at the LHC.  Indeed, this unexplored region of EW physics may provide clues to as yet unknown new physics processes at higher energy scales.  Using the 2012 LHC data recorded by the ATLAS experiment, we seek to make the first observation of one such EW process, the massive tri-boson final state: WWW.  It represents one of the first searches to probe the Standard Model WWWW coupling directly at a collider.  This search looks specifically at the channel where each W boson decays to a charged lepton and a neutrino, offering the best sensitivity for making such a measurement.  In addition to testing the Standard Model directly, we also use an effective field theory approach to test for the existence of anomalous quartic gauge couplings which could offer evidence for new physics at higher energies than those produced by the LHC.
\end{abstractpage}
