%11pt seems to be a good size, though it isn't specified as far as 
%I can tell --- my thesis got through the format review meeting in 11pt
\documentclass[11pt]{report}

%package for the bu thesis format
\usepackage{bu_math_thesis}

%put other packages you might need and custom declarations here
\usepackage{amsmath}

%%%%%%%%%%%%%%%%%%%%%%%%%%%%%%%%%%%%%%%%%%%%%%%%%%%%%%%%%%%%%%%%%%%%%%%%%%
\begin{document}


%%%%%%%%%%%%%%%%%%%%%%%%%%%%%%%%%%%%%%%%%%%%%%%%%%%%%%%%%%%%%%%%%%%%%%%%%%
%setup commands for the bu thesis style file

\title{Title}

\author{Author}

% Type of document prepared for this degree:
%   1 = Master of Science thesis,
%   2 = Doctor of Philisophy dissertation.
%   3 = Master of Science thesis and Doctor of Philisophy dissertation.
%   4 = Doctoral Dissertation Prospectus
\degree=1

\prevdegrees{Your previous degree, Your university, Year granted}

\department{Department of Mathematics and Statistics}

\university{Boston University}

\faculty{Graduate School of Arts and Sciences}

% Degree year is the year the diploma is expected, and defense year is
% the year the dissertation is written up and defended. Often, these
% will be the same, except for January graduation, when your defense
% will be in the fall of year X, and your graduation will be in
% January of year X+1
\defenseyear{2013}
\degreeyear{2013}

% For each reader, specify appropriate label {First, second, third},
% then name, then title. Warning: If you have more than five readers
% you are out of luck, because it will overflow to a new page.
% Sometimes you may wish to put part of the title in with the name
\reader{First}{First reader name, PhD}{title}
\reader{Second}{Second reader name, PhD}{title}
\reader{Third}{Third reader name, PhD}{title}

% The Major Professor is the same as the first reader, but must be
% specified again for the abstract page
\majorprof{Advisor name}{title}


%%%%%%%%%%%%%%%%%%%%%%%%%%%%%%%%%%%%%%%%%%%%%%%%%%%%%%%%%%%%%%%%%%%%%
% other set up commands which are a good idea

%the bottom margins should be ``as close as possible'' to 1 inch, so 
%allowdisplaybreaks is a good idea for theses with a lot of equations
\allowdisplaybreaks


%%%%%%%%%%%%%%%%%%%%%%%%%%%%%%%%%%%%%%%%%%%%%%%%%%%%%%%%%%%%%%%%%%%%%
%                       PRELIMINARY PAGES
% According to the BU guide the preliminary pages consist of:
% title, copyright (optional), approval,  acknowledgments (opt.),
% abstract, preface (opt.), Table of contents, List of tables (if
% any), List of illustrations (if any). The \tableofcontents,
% \listoffigures, and \listoftables commands can be used in the
% appropriate places. For other things like preface, do it manually
% with something like \newpage\section*{Preface}.

% This is an additional page (do not hand it in at the library) to print
% boxed-in title, author and degree statement so that they are visible through
% the opening in BU covers used for reports. This makes a nicely bound copy.

%\buecethesistitleboxpage

% Make the titlepage based on the above information.  If you need
% something special and can't use the standard form, you can specify
% the exact text of the titlepage yourself.  Put it in a titlepage
% environment and leave blank lines where you want vertical space.
% The spaces will be adjusted to fill the entire page.
\maketitle

% The copyright page is blank except for the notice at the bottom. You
% must provide your name in capitals.

%\copyrightpage

% Now include the approval page based on the readers information

\approvalpage

% The acknowledgment page should go here. Use something like
% \newpage\section*{Acknowledgments} followed by your text.

\newpage
\section*{Acknowledgments}

The style file is originally based on an MIT sytle file by Stephen
Gildea, modified for BU by Paolo Gaudiano.  Modified for CNS by
Jonathan Polimeni.  Further modifications by Janusz Konrad, Cameron
Morland, and Karen Yeats.

This example file takes some comments from a previous example file of
unknown authorship.

% The abstractpage environment sets up everything on the page except
% the text itself.  The title and other header material are put at the
% top of the page, and the supervisors are listed at the bottom.  A
% new page is begun both before and after.  Of course, an abstract may
% be more than one page itself.  If you need more control over the
% format of the page, you can use the abstract environment, which puts
% the word "Abstract" at the beginning and single spaces its text.

\begin{abstractpage}
This document serves as an example of how to use the BU thesis style
file.  This document is written from the point of view of a math grad student,
but the style file is agnostic on department.
\end{abstractpage}

% Now you can include a preface. Again, use something like
% \newpage\section*{Preface} followed by your text

% Table of contents comes after preface
\tableofcontents

% If you have tables, uncomment the following line
%\listoftables

% If you have figures, uncomment the following line
%\newpage\listoffigures

% List of Abbrevs is NOT optional (Martha Wellman likes all abbrevs
% listed)
% For mathematics a list of symbols is perhaps more appropriate, but
% fulfills the same role
% If your list is longer than one page, use the ``longtable'' package
% Just \usepackage{longtable} and then replace the word ``tabular'' 
% with ``longtable''.  You may have to download this package from
% CTAN or another internet source.
\chapter*{List of Symbols}
  \begin{tabular}{lp{0.75\textwidth}}
    1PI \dotfill & 1-particle irreducible, that is, 2-connected \\
    $\beta$ \dotfill & the physicists' $\beta$-function describing the
    nonlinearity of a Green function \\
  \end{tabular}


% END OF THE PRELIMINARY PAGES
\newpage
\endofprelim

%%%%%%%%%%%%%%%%%%%%%%%%%%%%%%%%%%%%%%%%%%%%%%%%%%%%%%%%%%%%%%%%%%%
% the body of the thesis goes here.

\chapter{Introduction}
The BU {\LaTeX} thesis style file has been around for some time.  Many
different versions have evolved.  Collecting together and charting the
heritage of the disperate versions could probably be a paper in a
suitable discipline.  This version has been through ECE and CNS and
has now arrived in math.

This version is an attempt to clean up one version in the hopes that
some of this fractuousness can be avoided in the future.  I believe it
to be correct.  Beyond correctness it has three main features.  Due to its
ECE heritage it has an option to make an unofficial title box page (to
match with the cut out in a cardboard report cover).  Also from ECE it
can be used to make prospectuses and other types of theses.  Finally
it includes some cv-making commands modified from a resume style file
by Miklos Csuros.

\chapter{Content}

\section{How to compile}

I confess that I don't know how to compile this in all the popular
programs everyone uses these days, but if you just put the style file
\verb+bu_math_thesis.sty+ wherever you usually do and the the rest
of the compilation is plain {\LaTeX} and plain bibtex for the example
bibliography.  You don't have to use bibtex to use this style file,
it's merely what this example uses (and a good idea for long or
complicated documents).

\section{Comments on features}

Chapters have no number on their first page and the numbers on
subsequent pages are in the middle at the top.  The latter is required
and the former is acceptable. 

\subsection{Subsections work}
\subsubsection{Subsubsections work}

The style file is compatible with many other styles and packages.  You
should have no problem including graphics in your favorite way for
example.  One should just be careful not to modify the headers or
margins unless one 
is sure one knows what one is doing.

Regarding margins, I have simply been able to compile this file (at
the command line in linux via \texttt{latex}) and then convert the dvi
to a ps file (at the command line via \texttt{dvips}) and print the ps
either on the printer \texttt{bott} or on \texttt{canon}.  I brought
a copy printed on \texttt{canon} to my format review meeting and
Martha Khan was happy with it though there is a small amount of
vertical jitter caused by the printer.  People have had problems in
the past with margins coming out differently than expected.  This may
partly be due to the fact that sometimes linux systems have A4 as the
default paper size which can cause problems when the resulting
document is inevitably printed on letter size paper.  If you are using
\texttt{dvips} on the command line, the way to make certain the result
is letter size is \texttt{dvips -t letter}.  That is what I do and it
works for me.

The bibliographic style can be whatever is standard in one's field.
For much of math this is the often maligned plain style, eg
\cite{bkerfc}.
 The style file has also been tested with apalike.  To
use apalike both set the \verb+\bibliographystyle{apalike}+ and
include the package \texttt{apalike}.   The bibliography can be single
spaced as below.  I'm using bibtex, though that's not required.

\chapter{Conclusions}
Have as many chapters as you like.  Good luck.

%%%%%%%%%%%%%%%%%%%%%%%%%%%%%%%%%%%%%%%%%%%%%%%%%%%%%%%%%%%%%%%%%%%%
% The back matter



% If you don't write the journal names out in full in the bibliography
% then you need a list of journal abbreviations
% If your list is longer than one page, use the ``longtable'' package
% Just \usepackage{longtable} and then replace the word ``tabular'' 
% with ``longtable''.  You may have to download this package from
% CTAN or another internet source.

\chapter*{List of Journal Abbreviations}
\begin{center}
\begin{tabular}{lp{0.56\textwidth}}
  Nucl. Phys. B \dotfill & Nuclear Physics B: Particle physics, field
  theory and statistical systems, physical mathematics \\
\end{tabular}
\end{center}

% The bibliography itself can be single spaced

\newpage
\singlespace
\bibliographystyle{plain}
\bibliography{www.bib}

%%%%%%%%%%%%%%%%%%%%%%%%%%%%%%%%%%%%%%%%%%%%%%%%%%%%%%%%%%%%%%%%%%
% finally you must include your cv.  You can do that whatever way you
% like including by formatting it in a totally different program.

% If you would like to grab it from some other source then be sure the
% page numbering is consecutive with the end of the bibliography and
% be sure it appears on the table of contents by adding a line such as
%  \addcontentsline{toc}{chapter}{Curriculum Vitae}

% If you would like to include it directly you could use the commands
% in the below example

\chapter*{Curriculum Vitae}

\end{document}
