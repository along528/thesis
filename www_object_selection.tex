\section{Physics Object Definition and Selection}
\label{sec:object_selection}
We attempt to identify and measure the particles coming from
the proton-proton collisions of the LHC by using the ATLAS detector.
The most interesting physics objects
for this analysis are the electrons and muons
that come from the $WWW$ decay. We also pay attention to 
the presence of hadronic activity and neutrinos, however, since these can
help discriminate the signal from the backgrounds.
Each type of particle has a unique signature in the detector
that allows us to identify the particle and reconstruct 
its properties, such as its charge and four-momentum. 
This reconstruction process does not guarantee
100\% accuracy either in identifying the particle or measuring its 
properties.
This process results in reconstructed ``physics objects''
which are selected using specific criteria optimized
for good identification efficiency and measurement resolution.
The selections used for the physics objects of interest are described below.

%define primary vertex here
%define isolation here? or elsewhere?
%do i need to talk about object energy corrections?
%do I need to say anything about the electron tight++ or muon tight id?

Muon objects are identified by the presence of tracks in both 
the ID and the MS that are shown 
to match using a statistical combination \cite{Hassani:2007cy}. 
After tight quality requirements, the performance of muon reconstruction
and identification is like in \cite{Aad:2014rra}.
To ensure that the track in the inner detector
indeed comes from a muon, requirements are placed
on the number of hits in the different sub-components of the inner detector.
%probably don't want to identify the MCP hit reqs, but I could...
The track is required to extrapolate back to the primary vertex 
to point within the boundaries of the MS and ID
within $|\eta|<2.5$.
The muon \pt~ at the primary vertex is chosen to be limited to $\pt > 10$~GeV.
We are not interested in 
muons coming from jets or other hadronic activity, therefore we
ask that they be isolated. The isolation of the muon is evaluated
in two ways: using tracks and using energy deposits in the calorimeter.
The isolation determined using tracks is calculated by adding
up the scalar sum of the \pt~ of all of the tracks (excluding
the muon track) in a cone of $\Delta R< 0.2$ from the muon track.
We ask that the isolation from tracks be less than 4\% of the muon \pt.
The isolation determined using the calorimeter is calculated in
a similar way except that energy deposits are used instead of tracks.
We then ask that the isolation from the energy deposits 
be less than 7\% of the muon \pt~when $\pt < 20$~GeV and 
less than 10\% of the muon \pt~otherwise. 





The signature for electron objects are that they have a track in the inner
detector that points to an energy deposit in the EM calorimeter.
Tight quality requirements are placed on the electrons to achieve
reconstruction and identification performance like in \cite{Aad:1744017}.
Similar to the muon objects, 
the electron track is required to extrapolate back to 
the primary vertex and have a $\pt>10$~GeV.
The direction of the electron energy 
deposits are also asked to fall within $|\eta| < 2.47$ and outside the 
transition region between the EM calorimeter 
barrel and endcap, $1.37 < |\eta| < 1.52$.
The electron objects are required to be isolated and have additional
requirements on the track extrapolation, similar to the muon objects.  
%what about further quality requirements?


Jet objects are associated with energy deposits in multiple 
neighboring cells of the electromagnetic and hadronic calorimeter systems.
They are reconstructed by grouping these cells
as topological clusters~\cite{Lampl:1099735}
using the anti-$k_t$ algorithm~\cite{Cacciari:2008gp} with $\Delta R < 0.4$.
The performance of jet identification in ATLAS is described in \cite{Aad:2016upy}.
The reconstructed jet objects are required to have 
$\pt>25$~GeV and $|\eta|<4.5$ so that
they are within the boundaries of the calorimeter systems.
The reconstructed jets are furthermore selected to suppress 
contamination from pileup events. This selection is performed by 
requiring that the majority of the
scalar sum of the \pt~of the tracks associated with the 
jet are also matched to the primary vertex. This is referred to
as the ``Jet Vertex Fraction''~\cite{Miller:1206864, ATLAS-CONF-2013-083}
and is only used for jets having $\pt < 50$~GeV and $|\eta|<2.4$, where
the algorithm is shown to perform well. Jets without any associated
tracks are always kept. 
%what about jet calibration?

It is also possible to identify jets that come from heavy flavor
decays, namely through the decays of $b$-hadron. We refer
to these as $b$-jets. A $b$-jet can frequently be identified 
because of the relatively long lifetime of the $b$ quark, which can 
result in a decay vertex that is displaced far enough
from the original primary vertex to be detected.
This can be used to ``tag'' jets as likely coming from
$b$ quarks. A multivariate $b$-tagging algorithm~\cite{ATLAS-CONF-2014-004}
is used with a working point determined to be 85\% efficient at
identifying $b$-jets. %what about the mis-tag rate?
Often, $b$-jets are associated with physics processes other than the signal
and are helpful in identifying background processes.
As a result, we choose to veto events where 
$b$-jets are present when looking in the signal regions.


The presence of neutrinos is inferred by a momentum
imbalance in the transverse plane, referred to as the missing
transverse energy or $\MET$. The $\MET$ is calculated by 
adding up all of 
the energy deposits from calorimeters cells within $|\eta| < 4.9$
and then calibrating them based on the the reconstructed
physics object they are associated with.
If the association is ambiguous then they are chosen based on the following
preference (from most preferred to least): electrons, photons, 
hadronically decaying $\tau$-leptons, jets, and muons.
If the calorimeter deposit is not associated with any physics object
they are still considered using their own calibration.
The sum is modified to take into account the momentum of muons,
which typically leave minimum ionizing energy deposits
in the calorimeter without being completely stopped.



It is possible that the reconstructed electrons, muons, and jets
may overlap with each other inside the detector.  This can occur
because because of the same physics object being reconstructed as different
objects in the ATLAS detector.  We handle these occurrences using the following
scheme in order of precedence:
\begin{enumerate}
	\item Electron-Muon Overlap: If $|\Delta R(e,\mu)| < 0.1$, 
	then keep the muon and throw away the electron.
	\item Electron-Jet Overlap: If $|\Delta R(e,j)| < 0.2$, 
	then keep the electron and throw away the jet.
	\item Muon-Jet Overlap: If $|\Delta R(\mu,j)| < 0.2$, 
	then keep the muon and throw away the jet.
\end{enumerate}
The direction is taken from the calorimeter information for electrons,
from the combined track information for muons, and from the anti-$k_{\mathrm{T}}$
algorithm for jets.
No momentum smearing or calibration corrections
are applied to the reconstructed object directions. 
Using this scheme means that a precedence is set when 
reconstructed objects overlap such that $\mu > e > j$ where ``$>$'' should
be interpreted to mean ``is kept instead of''. 

The motivation for this scheme
is as follows. Muons will frequently radiate photons which then can pair-produce
to electrons.  If the energy of one of the pair-produced electrons is 
large enough then this can be reconstructed as well and will likely be collimated
with the muon.  Since the electron comes from the muon radiation and
since the reverse process with an electron having pair-produced muons
is heavily suppressed, the muon is kept preferentially.  The reconstruction
of overlapping electrons and jets
would rely on much of the same energy deposits in the calorimeter.  But the electron
reconstruction also relies on matching with a well defined inner detector
track.  It is thus assumed that if an electron overlaps with a reconstructed
jet that this is more likely to be the signature of a high energy electron.
Finally, if a muon overlaps with a jet, the muon could come from a heavy flavor 
decay. If this occurs, we choose to keep the event and consider only the muon.


