\section{Physics Object Definition and Selection}
\label{sec:object_selection}
We attempt to identify the stable particles coming from
the proton-proton collisions of the LHC by using the ATLAS detector.
The most interesting physics objects
to this analysis are the electrons and muons
that come from the $WWW$ decay. However, we also pay attention to 
the presence of hadronic activity and neutrinos, since these can
help discriminate the signal from the backgrounds.
Each type of particle has a unique signature in the detector
that allows us to identify the particle and to reconstruct 
its properties, such as its charge and four-momentum. 
This reconstruction process does not guarantee
100\% accuracy either in identifying the particle or measuring its 
properties. As such, the reconstruction process results in reconstructed
``physics objects'' that may or may not map accurately 
to the underlying particle or physics it is trying to describe. That 
being said, this mapping is usually very successful due to the high quality
of the detector and the design of the reconstruction algorithms used.
To maximize the success of reconstruction we look at physics
objects selected only where the reconstruction is well understood.
The selections used for the physics objects of interest are described below.

%define primary vertex here
%define isolation here? or elsewhere?
%do i need to talk about object energy corrections?
%do I need to say anything about the electron tight++ or muon tight id?

Muon objects are identified by the presence of tracks in both 
the ID and the MS that are shown 
to match using an extrapolation process through the gap between the
two sub-detectors. To ensure that the track in the inner detector
indeed comes from a Muon, strict requirements are placed
on the number of hits in the different sub-components of the inner detector.
%probably don't want to identify the MCP hit reqs, but I could...
The track is extrapolated back to the primary vertex and is forced
to be pointing within the boundaries of the MS and ID
by requiring that $|\eta|<2.5$.
The muon \pt~ at the primary vertex is chosen to be limited to $\pt > 10$~GeV
where there is adequate momentum resolution. We are not interested in 
muons coming from jets or other hadronic activity, therefore we
ask that they be isolated. The isolation of the muon is evaluated
in two ways: using tracks and using calorimeter deposits.
The isolation determined using tracks is calculated by adding
up the scalar sum of the \pt~ of all of the tracks (excluding
the muon track) in a cone of $\Delta R< 0.2$ from the muon track.
We ask that the isolation from tracks be less than 4\% of the muon \pt.
The isolation determined using calorimeter deposits is calculated in
a similar way except that calorimeter deposits are used instead of tracks.
We then ask that the isolation from calorimeter deposits 
be less than 7\% of the muon \pt~when $\pt < 20$~GeV and 
less than 10\% of the muon \pt~otherwise. Additional requirements
are placed upon the track extrapolation to ensure that it comes from
the primary vertex.


The signature for electron objects are that they have a track in the inner
detector that points to an energy deposit in the EM calorimeter.
The electron at the primary vertex is expected to have $\pt>10$~GeV, similar
to the muon objects. The direction of the electron energy 
deposits are also asked to fall within $|\eta| < 2.47$ and outside the 
transition region between the EM calorimeter barrel and endcap, $1.37 < |\eta| < 1.52$.
The electron objects are required to be isolated and have additional
requirements on the track extrapolation the same way as for muon objects.  
%what about further quality requirements?


Jet objects are associated with energy deposits in multiple 
neighboring cells of the EM and hadronic calorimeter systems.
Jet objects are reconstructed by grouping these cells
as topological clusters~\cite{Lampl:1099735}
using the anti-$k_t$ algorithm~\cite{Cacciari:2008gp} with $\Delta R < 0.4$.
The reconstructed jet objects are required to have a reconstructed
$\pt>25$~GeV and to have $|\eta|<4.5$ so that
they are within the boundaries of the calorimeter systems.
The reconstructed jets are furthermore selected to suppress the likelihood that
they come from pileup events. This selection is performed by 
requiring that the majority of the
scalar sum of the \pt~ of the tracks associated with the 
jet are also matched to the primary vertex. This is referred to
as the so-called ``Jet Vertex Fraction''~\cite{Miller:1206864, ATLAS-CONF-2013-083}
and is only used with jets with $\pt < 50$~GeV and $|\eta|<2.4$ where
the algorithm is shown to perform well. Jets without any associated
tracks are always kept. 
%what about jet calibration?

It is also possible to identify jets that come from heavy flavor
decays, namely $b$ quark and $b$-hadron decays. We refer
to these as $b$-jets. $b$-jets can frequently be identified 
because of the relatively long lifetime of the $b$ quark, which can 
result in a decay vertex that is displaced from the original primary vertex.
This can be taken advantage of to ``tag'' jets as likely coming from
$b$ quarks. A multivariate $b$-tagging algorithm~\cite{ATLAS-CONF-2014-004}
is used with working point determined to be 85\% efficient at
identifying $b$-jets. %what about the mis-tag rate?
$b$-jets are associated with physics processes other than the signal
and are helpful in identifying background processes.
As a result, we choose to veto events where 
$b$-jets are present when looking in the signal regions.


The presence of neutrinos are inferred by a momentum
imbalance in the transverse plane, referred to as the missing
transverse energy or $\MET$. The $\MET$ is calclulated by 
adding up all of 
the energy deposits from calorimeters cells within $|\eta| < 4.9$
and then calibrating them based on the the reconstructed
physics object they are associated with.
If the association is ambiguous then they are chosen based on the following
preference (from most preferred to least): electron, photons, 
hadronically decaying $\tau$-leptons, jets, and muons.
If the calorimeter deposit is not associated with any physics object
they are still considered using their own calibration.
The sum is modified to take into account the momentum of muons,
which typically leave trace energy deposits
in the calorimeter without being completely stopped.



It is possible that the reconstructed electrons, muons, and/or jets
may overlap with each other inside the detector.  This can occur
because because of the same physics object being reconstructed as different
objects in the ATLAS detector.  We handle these occurrences using the following
scheme in order of precedence:
\begin{itemize}
	\item Electron-Muon Overlap: If$|\Delta R(e,\mu)| < 0.1$ then the  muon is kept while the electron is thrown away.
	\item Electron-Jet Overlap: If $|\Delta R(e,j)| < 0.2$ keep the electron and throw away the jet.
	\item Muon-Jet Overlap: If $|\Delta R(\mu,j)| < 0.2$ keep the muon and throw away the jet
\end{itemize}
For electrons, the direction is taken only from the electron calorimeter
information.  Muons use the full combined track information while jets
use the direction taken from the anti-k$_{\mathrm{T}}$  algorithm.
No momentum smearing or calibration corrections
are applied to the reconstructed object directions. 
Using this scheme means that a precedence is set when 
reconstructed objects overlap such that $\mu > e > j$ where '$>$' should
be interpreted to mean 'is kept instead of'. The motivation for this scheme
is as follows. Muons will frequently radiate photons which then can pair-produce
to electrons.  If the energy of one of the pair-produced electrons is 
large enough then this can be reconstructed as well and will likely be collimated
with the muon.  Since the electron comes from the muon radiation and
since the reverse process with an electron having pair-produced muons
is heavily suppressed, the muon is kept preferentially.  The reconstruction
of overlapping electrons and jets
would rely on much of the same calorimeter energy deposits.  But the electron
reconstruction also relies on matching with a well defined inner detector
track.  It is thus assumed that if an electron overlaps with a reconstructed
jet that this is more likely to be the signature of a high energy electron.
Finally, if a muon overlaps with a jet, the muon could come from a heavy flavor 
decay. If this occurs, we choose to keep the event and consider only the muon.


