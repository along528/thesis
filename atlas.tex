%step through each section of the atlas detector
%could spend more time on the muon spectrometer (like the new EE A-side detectors)
%and more time on the muon trigger
%Clare put in the minimal detail (no more than 20 pages) while Jeremy had 50 pages
%I could maybe fit somewhere in between

%need to define Muon Spectrometer (MS)
%need to define Inner Detector (ID)
%Assumptions:
%Will describe trigger


%I need to define things here like eta, pt, deltaR, etc 
%I may need to talk about isolation here
The ATLAS detector~\cite{ATLAS} is designed to measure
the products of the particle collisions produced by the LHC.
In particular, the detector seeks to measure those stable 
(or metastable) particles whose decay lifetime is sufficiently
long enough to interact with the detector.  This includes
a variety of fundamental particles (like muons) as well as 
composite particles (like neutrons). The wide variety of 
particles to be measured requires the implementation
of several sub-detector systems that work in tandem 
to identify and measure the properties of these particles.
The products of the LHC collisions travel in all directions, 
thus if there is any hope to measure all of the products of the collision
the detector must completely surround the collision point. 
The symmetry of the LHC dictates a cylindrically symmetric
shape for the ATLAS detector around this point.
There is another reason to build the detector to enclose
the collision point, since 
even with such a detector, the vanishingly small
cross-section of the neutrinos, which are also products
of the collision, means that these ghostly particles
will inevitably go undetected. But there is still hope!
If all other particles are measured, 
the presence of the neutrinos can be inferred 
using momentum constraints, as will be discussed later.  
In this way, the ATLAS detector is most generally suited
for measuring the wide variety of processes produced
by the LHC. 

\begin{figure}[ht!]
\centering
\includegraphics[width=.9\textwidth]{figures/atlas/detector.jpg}
\caption{A diagram of the ATLAS detector where the detector has
been artifically opened up to reveal the LHC beam line and the
various sub-detector components within. The sub-detector components
are labelled as such.}
\label{fig:atlas}
\end{figure}

A diagram of the ATLAS detector can be seen in \fig\ref{fig:atlas}.
It has a clearly cylindrical shape with a diameter of 25 meters
and length of 44 meters. The detector is massive, weighing
in at roughly 7000 tons; but it is also highly granular, with
over 100 million electronic channels that are arranged very precisely, 
in many cases on the order of tens of microns.
In the ``opened'' view of \fig\ref{fig:atlas}, the proton-proton
collisions from the LHC occur at the center core of the detector
and the sub-detector components build up around this point.
The detectable products of the collision pass outward from the collision
point through the different components where their energy and momentum
are measured. The way in which the particles interact with the various
sub-detector systems helps to identify the types of 
particles produced.
This can be more clearly seen in the diagram of 
\fig\ref{fig:atlas_wedge}, which shows how the most predominant
products of the LHC collisions interact with the different
components of the ATLAS detector.
Nearest the collision point is the inner detector (ID), designed to 
measure the paths of charged particles passing through using several
different subsystems. This 
is surrounded by a 2 Tesla solenoidal magnetic field which 
bends the trajectory of the charged particles by an amount
inversely proportional to the particle momentum, thereby allowing
for a momentum measurement.  Beyond that is the calorimeter system
which measures the energy deposits of all particles passing 
through (except for muons and neutrinos) by stopping them in their 
tracks. The calorimeter system 
itself is divided up into components which fall into two main 
categories: the electromagnetic (EMCAL)
and hadronic calorimeter (HCAL) systems.
The EMCAL is situated in front of the HCAL and is designed
primarily to stop fundamental particles like electrons and photons. 
The HCAL then is designed to stop
composite particles like protons and neutrons through.
Surrounding the carlorimeter system is the muon spectrometer (MS),
which is the largest component of the ATLAS detector and the one
that determines its size. It is designed to measure the 
trajectory of muons through ionization as they pass through
and ultimately leave the detector. The MS is also composed of 
three large superconducting air-core toroid magnets with an 
average magnetic field of roughly 0.5 Tesla which allows for
a measurement of the muon momentum. The neutrinos 
pass through undetected.

\begin{figure}[ht!]
\centering
\includegraphics[width=.9\textwidth]{figures/atlas/wedge.jpg}
\caption{A diagram of one wedge of the ATLAS detector
as viewed from looking down the beam line. 
The sub-detector components are shown along with the 
particles that most predominately come from the collision.
The paths of the particles indicate how each particle typically 
interacts with the detector.}
\label{fig:atlas_wedge}
\end{figure}

The geometry of the ATLAS detector is defined using a 
right-handed cylindrical coordinate system with the $x$-axis
pointing inwards towards the center of the LHC ring, the $y$-axis point
up, and the $z$-axis pointing along the beamline.
The $x$ vs $y$ plane, which is perpendicular to the beamline,
is referred to as the transverse plane and typically 
defined using cylindrical coordinates with $r$ being the distance
from the center and $r$ being the azimuthal angle.
For describing the direction of the particle with respect 
to the beamline, a quantity called the pseudo-rapidity, $\eta$, is commonly
used
\begin{equation}
\eta = -\ln \tan (\theta/2) 
\label{eq:pseudorapidity}
\end{equation}
which is a function only of the polar angle, $\theta$, which 
itself is defined
as the direction of the particle with respect to the positive $z$-axis.
Changes in the pseudorapidity are 
invariant under Lorentz transformations along
the $z$-axis in the limit that the particle is approximately massless. 
As a result, its distribution is typically flat over
a wide range of pseudorapidity. At the LHC, most stable particles 
are produced with energies much larger than their mass, making the 
massless approximation valid.
The ATLAS detector has nearly uniform $2\pi$ coverage in $\phi$,
while in $\eta$ the ID is restricted to $|\eta| < 2.5$,
the MS to $|\eta| < 2.7$, and the calorimeter systems all the way
to $|\eta| < 4.9$.

The total momentum of the proton-proton collision in the transverse
plane is zero. Since the detector has full azimuthal coverage 
in the transverse plane, we can test this constraint by measuring
the total momentum from the particles measured in the detector.
The magnitude of a particle's momentum 
in the transverse plane is referred to as the 
transverse momentum, $\pt$.
Thus, we may refer to this constraint as
\begin{equation}
\Bigg| \sum_{i\in\textrm{All Particles}} \vec{p}_{\textrm{T},i} \Bigg| = 0
\end{equation}
where the transverse momentum is added vectorially and then
the magnitude is taken.
After adding up the $\pt$ of all of the particles to obtain
the total transverse momentum, 
any imbalance with respect to this constraint
is referred to as the
missing transverse energy, $\met$, and is attributed to the 
neutrinos produced in the collision. 
There is no such constraint on the momentum along the $z$-direction
because the momentum fraction of the partons as taken
from the PDFs are not known with certainty. This is the case
even though the momentum along the $z$-direction of the 
protons from which the partons are taken is, in fact, known.
Thus there is no way of determining with certainty the 
momentum of the neutrinos in the $z$-direction.



\section{Inner Detector}
\label{sec:atlas_id}

\begin{figure}[ht!]
\centering
\includegraphics[width=0.7\textwidth]{figures/atlas/id_barrel.eps}
\caption{Diagram of the ATLAS Inner Detector (ID) system showing 
a wedge of the barrel system.  The three detector systems
are clearly labeled. The LHC beam pipe runs parrel to the system
and is shown at the bottom of the diagram.}
\label{fig:atlas_id_barrel}
\end{figure}

\begin{figure}[ht!]
\centering
\includegraphics[width=0.7\textwidth]{figures/atlas/id_endcap.eps}
\caption{Diagram of the ATLAS Inner Detector (ID) system showing 
a wedge of the endcap system as well as a part of the SCT and Pixel
barrel systems.  The detector systems
are clearly labeled.  The LHC beam pipe runs parrel to the system
but is not shown. Trajectories of two charged tracks 
with a $\pt=10\GeV$ are shown along $\eta=1.4$ and $\eta=2.2$ are shown 
byt the solid bright red lines.}
\label{fig:atlas_id_endcap}
\end{figure}

The inner detector (ID) is the 
detector system that is closest to the beam pipe and thus
first system that the products of the LHC collisions encounter
on their way from the collision point. Its primary role is 
to measure the trajectory and momentum of charged particles
through ionization as they pass through the detector.
It must be capable of measuring these tracks with high precision
in order to obtain precise momentum measurements and also to be able
to accurately extrapolate the tracks back to the collision point
to obtain primary and secondary interaction vertices. In addition,
since the system is so close to the LHC beam line, it
must be able to handle the high particle fluxes. This requires that
the ID must have a very high granularity and fast electronics
readouts such that the occupancy of the
detector is small enough to distinguish individual tracks. In addition,
the detector materials and electronics must be sufficiently radiation
hard that they can withstand years of LHC exposure time.
These tough requirements push the limits of available technology and thus
make the ID the most sophisticated detector system in ATLAS.



\begin{figure}[ht!]
\centering
\includegraphics[width=0.7\textwidth]{figures/atlas/pixel.eps}
\caption{A cut-out diagram of the ATLAS pixlel detector showing 
the arrangement of the pixel modules (green) in three layers of the barrel
and three layers of one end-cap system. Some of the support structure is
also shown.}
\label{fig:atlas_pixel}
\end{figure}

There are three different detector subsystems within the ID, together
immersed in a roughly uniform 2 Tesla axial magnetic field: the pixel detector,
the silicon microstrip (SCT) detector, and the transition radiation
tracker (TRT). These three detector systems can be seen 
in the barrel in \fig\ref{fig:atlas_id_barrel} and from an alternate
view also showing one of the end-caps in \fig\ref{fig:atlas_id_endcap}. 
The pixel detector
is composed of more than seventeen hunderd thin doped silicon sensors with 
dimension $19\times 64~\textrm{mm}^2$. Each sensor has more than forty-six
thousand readout channels,
corresponding to the ``pixels'' which give the detector its name. 
A charged particle passing through an individual pixel can send a signal
which identifies the location of a charged particle. The combination of 
several layers can thus be used to form the trajectory of the particle. %sagitta
Each sensor is attached to a single readout electronic board, which comprises
one module.
The modules are arranged into three cylinrical barrel layers and 
two end-caps each with three disk-shaped layers such that there is uniform
azimuthal coverage. A cut-out diagram of the pixel detector 
structure with modules in place in both the barrel and end-caps is shown 
in \fig\ref{fig:atlas_pixel}. The barrel covers roughly 
$|\eta|<1.7$ and the two end-caps roughly $1.7<|\eta|<2.5$.
Test beam measurements show that the 
spatial resolution of the pixel detector is around $12~\mu\textrm{m}$ in 
the $R-\phi$ plane and is slightly degraded orthogonal to this plane.



The SCT uses almost sixteen thousand thin silicon strip sensors, though not of the 
same type as in the pixel detector. 
A barrel silicon strip sensor has dimension $6.36\times 6.40~\textrm{cm}^2$
with 768 readout strips running along the longer dimension. The barrel
strips are placed in four concentric cylindrical layers, uniformly in azimuth,
with the strips aligned axially, and covering roughly
$|\eta|<1.4$, as can be seen in \fig\ref{fig:atlas_id_barrel}.
In each of the two end-caps the sensors are made to form nine
disks spaced apart along the axial 
direction, covering roughly $1.4 < |\eta|<2.5$, 
seen in \fig\ref{fig:atlas_id_endcap}. The strips are similar
to the barrel except that they are tapered along the strip direction.
The sensors are then oriented such that the taper expands radially outward.
In a test beam, the spatial resolution is found to be about $16~\mu\textrm{m}$
in the $R-\phi$ plane. Due to the length of the strips, the precision is considerably
worse in the axial direction for the barrel and the radial direction for 
the end-caps, with a precision of roughly $580~\mu\textrm{m}$.


The TRT uses a fundamentally different technology 
than the pixel and SCT.
Drift tubes are used of 4~mm in diameter 
which are filled with a Xenon-based gas mixture
and with an anode wire running through the center.
The advantage of using these tubes is that they can be placed in close
proximity such that many measurements, around 36,
can be made on a single charged track. Another important feature
of the TRT is its ability to identify electrons using transition radiation.
This is achieved by surrounding the tubes in polypropylene material to induce
transition radiation from incident electrons and taking advantage
of the discrimination power of the Xenon-based gas between 
transition radiation and tracking signals.
For electrons with $\pt>2\GeV$, usually 7 to 10 hits due to transition
radiation will be measured.
The barrel TRT runs from rougly $|\eta|<0.7$ and
is constructed from 144 cm long straws with two straws aligned
axially back-to-back. Over fifty-two thousand straws are interleaved
with polypropylene fibers to form 73 layers of straws surrounding the beampipe with
a cylindrical symmetry and uniform coverage in azimuth,
seen in \fig\ref{fig:atlas_id_barrel}.
In each of the two end-caps, two wheels are formed from over seventy-three
thousand straw tubes, 37 cm in length, 
oriented oriented and distributed uniformly in azimuth. 
The inner wheel is formed from twelve layers and the outer wheel from eight
layers with 768 straws in each layer, seen in \fig\ref{fig:atlas_id_endcap}.
The end-caps cover roughly $0.7<|\eta|<2.2$.
An individual straw has a precision of about $170~\mu\textrm{m}$ along
its diameter.




efficiency of electron identification?


large number of readouts?
occupancy??
noise??
sagitta???
explain how the b-field is used?
eta coverage?
materials?
electronics?
more figures?



\section{Calorimeters }
\begin{figure}[ht!]
\centering
\includegraphics[width=0.9\textwidth]{figures/atlas/calorimeter.eps}
\caption{Diagram of ATLAS calorimeter system with cut-out portion
to allow a view of the nested sub-components.}
\label{fig:atlas_calorimeter}
\end{figure}

The ATLAS calorimeter is designed to measure the energy
deposits of the products of the LHC collisions which pass through
it except for muons and neutrinos.  A diagram of the 
calorimeter system can be seen in \fig\ref{fig:atlas_calorimeter}.

\begin{figure}[ht]
\centering
\includegraphics[width=.5\textwidth]{figures/atlas/emcal_accordion.png}
\caption{Photo of three EMCAL sampling layers
showing the ``accordion'' structure. In the picture, 
the horizontal directions corresponds to 
the radial direction when the detector is in position, which is
the direction the LHC products would follow.}
\label{fig:atlas_emcal_accordion}
\end{figure}

\begin{figure}[ht]
\centering
\includegraphics[width=.8\textwidth]{figures/atlas/emcal_barrel_module.eps}
\caption{ A diagram of one EMCAL barrel module 
covering $22.5^{\circ}$ in azimuth.}
\label{fig:atlas_emcal_module}
\end{figure}

The calorimeter system is split into three main systems, 
the electromagnetic calorimeter (EMCAL), the 
tile hadronic calorimeter (HCAL), the hadronic 
end-cap calorimeter (HEC), 
and the Forward Calorimeter (FCAL).
The EMCAL is a sampling calorimeter that uses lead as the sampling
medium and liquid Argon (LAr) as
the active medium from which the charge of the electromagnetic
shower produced by incident particles on the sampling medium
can be collected.  LAr is used as the active medium 
because of its radiation hardness and its linear response.
%as evidenced in \fig\ref{fig:atlas_emcal_response}.
The lead sampling medium alternates with the active LAr medium
using lead plates 1-2~mm thick with an approximately 4~mm 
LAr gap between each sheet and electrodes placed in the middle of
the gaps.
The lead sheets are constructed using a unique ``accordion'' structure,
as seen in \fig\ref{fig:atlas_emcal_accordion}. 
This is to provide a uniform resolution with no gaps
in the azimuthal direction.
The EMCAL itself can be split up into a barrel region ranging
from $0<|\eta|<1.3$ and two end-cap regions ranging from 
$1.5 < |\eta| < 3.2$.
The thickness of the barrel region has a radiation length, \xzero,
ranging from $22~\xzero$ to $30~\xzero$
for $|\eta|<0.8$ and from $24~\xzero$ to $33~\xzero$ for
$0.8 < |\eta| < 1.3$.
The barrel region is divided into individual modules
which together surround the beamline
in a cylindrical shape.  A diagram of one such module
can be seen in \fig\ref{fig:atlas_emcal_module}.
From this one can see that each module is segmented in $\eta$
and $\phi$, as well as into three layers in depth.
The segmentation is applied to obtain pointing inofrmation, 
which aids in the identification and measurement of electromagnetic
objects in conjunction with measurements from the ID.
The very fine segmentation in $\eta$ of the first layer
in depth is important for precision tracking measurements.
The second layer has a larger depth and thus collects most of the energy.
There are two identical end-cap regions, one on each side of the 
collision point. Each end-cap region consists of two wheels: the 
outer wheel from $1.4756 < |\eta| < 2.5$, with a thickness ranging from
$24~\xzero$ to $38~\xzero$, and 
the inner wheel from $2.5 < |\eta| < 3.2$, with a thickness ranging from
$26~\xzero$ to $36~\xzero$.
The regions from $1.5 < |\eta|<2.5$ in the inner and outer wheels both
have three layers, with the first being a finely segmented precision
layer similar to the barrel regions. Outside this region there 
are only two layers with a coarser segmentation.
The EMCAL also consists of a presampler detector with a single layer of LAr in 
front of the full barrel EMCAL and in front of the end-cap EMCAL calorimeters
from $1.5 < |\eta| < 1.8$; this aids in the measurement of the energy deposits
prior to reaching the EMCAL and for a better understanding of the energy deposited
in the transition region between the barrel and end-caps.


\begin{figure}[ht]
\centering
\includegraphics[width=.6\textwidth]{figures/atlas/hcal_module.eps}
\caption{ A diagram of one tile HCAL module 
covering $5.625^{\circ}$ in azimuth. The radial direction when 
positioned in the detector corresponds to the vertical direction in the
image.}
\label{fig:atlas_hcal_module}
\end{figure}


The tile HCAL is a steel sampling calorimeter with scintillating tiles used as the 
active material.
Steel is chosen as the sampling material since it gives a good
depth in interaction lengths, $\lambda$, with a maximum depth
of $7.4~\lambda$, while also having a low cost.
It is split into a central barrel and two extended barrels 
which together cover a 
region from $|\eta|< 1.7$, as can be seen in \fig\ref{fig:atlas_calorimeter}.
Same as in the EMCAL barrel, the tile HCAL is divided into individual modules
that surround the collision point in azimuth; a diagram of one such
module is shown in \fig\ref{fig:atlas_hcal_module}.
The scintillating tiles alternate periodically with the self-supporting
steel body and are oriented radially. 
The scintillation
light is routed through wavelength-shifting fibers and collected
at photomultiplier tubes placed at the back of the module.
This configuration allows for a near uniform coverage in azimuth.  
In the crack region from $1.2 < |\eta| < 1.6$ between the central barrel 
and extended barrels, special modules are placed to recover and correct
for energy losses in this region.

\begin{figure}[ht] 
\centering
\includegraphics[width=.7\textwidth]{figures/atlas/hec.pdf}
\caption{ A schematic showing one quadrant of the 
HEC system in the $R$-$z$ plane. The dashed lines indicate
the pointing direction achieved by the segmentation of the 
readouts.  Dimensions are in mm.  }
\label{fig:atlas_hec}
\end{figure}


The HEC is designed to measure hadronic energy deposits in the 
end-cap regions from $1.5 < |\eta|< 3.2$. It uses 
copper plates as the sampling material with LAr gaps
for the active material. Two separate wheels are formed from
flat plates of copper alternating with LAr gaps further divided by electrodes for
collecting the ionization charge from the hadronic shower in the LAr.
The rear wheel is more coarse than front wheel;
This can be seen in the schematic of \fig\ref{fig:atlas_hec}.
The electronics readout is segmented such that pointing information
can be obtained, as indicated by the dashed lines.
The maximum radial depth of the HEC is roughly $10~\lambda$.



\begin{figure}[ht] 
\centering
\includegraphics[width=.7\textwidth]{figures/atlas/fcal.eps}
\caption{ A schematic showing the end-cap of the EMCAL,
the two HEC modules, and the three FCAL modules, as well
as additional shielding, in one
quadrant of the ATLAS detector as viewed 
in the $R$-$z$ plane.
The $R$-direction is shown with a larger scale than in the $Z$-direction.
}
\label{fig:atlas_fcal}
\end{figure}


The FCAL is in the region of the detector nearest to the beamline, 
where the radiation flux is highest, covering the range
from $3.1 < |\eta| < 4.9$. It is split into three cylindrical modules,
oriented as in \fig\ref{fig:atlas_fcal}, with the first 
being designed for measuring electromagnetic deposits and the other
two for hadronic deposits.
Each FCAL module is constructed from copper plates with roughly ten thousand
uniformly spaced holes drilled in the direction parallel to the beamline.
The holes are filled with rods serving as the primary sampling material, 
with a thin LAr gap surrounding the rods serving as the active material.
In the first FCAL, optimized for electromagnetic deposits,  
copper rods are used, while in the second and third FCAL modules, 
the rods are made from tungsten, which has a higher interaction length.
The first FCAL has a radiation length of $27.6~\xzero$ and an 
interaction length of $2.66~\lambda$. Meanwhile, the interaction
length of the second and third modules is around $3.6~\lambda$.



Shielding? Resolution and response?




\section{Muon Spectrometer}
%do i want to talk about muon reconstruction?
%definitely mention muon resolution
\section{Trigger}

