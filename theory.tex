
%Here I need to go through and describe all of the things that are important for the discussion
%aim for 20 pages

%things to discuss include:
%- A general introduction to the standard model (probably no more than a page or two)
%- Electroweak physics
%- W decay
%- parton distribution functions (maybe with some thorough discussion on this) as well as renormalization and factorization scales

%- WWW production (do I actually want to discuss this at all here??)
%- Effective field theory and anomalous coupling measurements


%assuming these will be defined here
% Quartic Gauge Couplings (QGC) 
% Standard Model (SM)
% anomalous Quartic Gauge Couplings (aQGC)
% Parton Distribution Function (PDF)


\section{The Standard Model}

The Standard Model (SM) is a theory which describes all of the
observed matter and interactions in the universe, except for gravity.
It is built from a quantum field theory where the consitutent particles
and interactions fit into a non-abelian 
$\suthree\times\sutwo\times\uone$~gauge symmetry.
From these symmetries come the matter fermions, split 
into the quarks and leptons, and the force-carrying bosons
that mediate their interactions.
The \suthree~symmetry describes the theory of Quantum Chromodynamics (QCD)
which explains the interaction of the quarks via the gluons, the
gauge bosons that mediate the strong force.
The remaining $\sutwo\times\uone$~symmetry 
describes the Electroweak (EW) theory which explains
the interactions of the quarks and leptons via the 
electroweak gauge bosons that mediate
the electroweak force: $W$, $Z$, and $\gamma$ (i.e. the photon).
The EW theory is itself a unified description of the weak force,
involving the $W$ and $Z$, and the electromagnetic force, involving
just the photon.
The $W$ and $Z$ gauge bosons (as well as the quarks and leptons) receive
their non-zero masses through the process of electroweak symmetry
breaking (EWSB). The simplest form of EWSB
introduces an additional 'Higgs' field that
predicts a single new fundamental scalar boson. This boson is the
famous Higgs boson which was discovered recently at the LHC, thereby
confirming this last component of the SM.
%cite

All of the 
observed fundamental 
matter particles in the universe are described by the quark
and leptons of the SM. Their properties are listed 
in \tab\ref{tab:theory_fermions}. 
The particles can be distinguished
by their charges and their masses.
The charges describe how (and if) the particles participate in
different interactions.
Those fermions with electric charge (all but the neutrinos) 
participate in the electromagnetic
interactions. The quarks have color charge (sometimes 
just called color), which allows them to 
participate in the QCD interactions. All fermions also participate in 
the weak interactions. The types of allowed weak interactions are determined
by a combination of the electric charge as well as the weak isospin
and weak hypercharge, described later.
%explain better
Since all of the particles are fermions, they all have a spin of 1/2.
The masses of the particles are not predicted by the theory, but 
are essential for understanding their stability and decay properties
as well as their kinematic behavior.
Each particle also has a corresponding anti-particle with the same mass but
whose electric charge has opposite sign. The neutrinos, with zero 
electric charge, could possibly be their own 
anti-particle (so-called Majorana fermions), but
this has yet to be confirmed.

\begin{table}[ht]
\centering
\small
\begin{tabular}{|lclc|c|c|}
\hline
                        & Generation & Name & Symbol  & Charge & Mass [MeV]\\
\hline
\multirow{6}{*}{Quarks}& \multirow{2}{*}{First}& Up & $u$   &  $~~2/3$    &  $2.3 ^{+0.7}_{-0.5}$    \\
                       & & Down & $d$  &  $-1/3$    &  $4.8^{+0.5}_{-0.3}$    \\
		       \cline{2-6} 
                       &\multirow{2}{*}{Second}& Charm & $c$   & $~~2/3$     & $1275 \pm 25$     \\
                       && Strange & $s$   & $-1/3$     & $95 \pm 5$     \\
		       \cline{2-6} 
                       &\multirow{2}{*}{Third}& Top & $t$   &  $~~2/3$    &  $173210 \pm 874$    \\
                       && Bottom & $b$   &  $-1/3$    & $4180 \pm 30$     \\
\hline
\hline
\multirow{6}{*}{Leptons}&\multirow{2}{*}{First} &Electron & $e$     &  -1    &  $0.510998928 \pm 0.000000011 $ \\
                        & & Electron Neutrino & $\nu_e$  & 0    &   $ < 0.002$    \\
		       \cline{2-6} 
                        &\multirow{2}{*}{Second} &Muon & $\mu$     &   -1   &   $105.6583715 \pm 0.0000035$   \\
                        & &Muon Neutrino & $\nu_{\mu}$  & 0   &  $ < 0.19$    \\
		       \cline{2-6} 
                        & \multirow{2}{*}{Third}&Tau & $\tau$     &   -1   &  $1776.86 \pm 0.12$    \\
                        & &Tau Neutrino & $\nu_{\tau}$  & 0    &  $< 18.2 $    \\
\hline
\end{tabular}
\label{tab:theory_fermions}
\caption{Summary of the electric charge and measured masses of the SM fermions. Mass measurements
are taken from the Particle Data Group \cite{PDG:2014} and 
are shown to the best precision available with their measured uncertainties.
Particles are also organized by their generation.
The bottom quark mass measurement is shown using the $\overline{\textrm{MS}}$ 
renormalization scheme.
The top quark mass uncertainty combines the reported statistical and systematic uncertainties 
in quadrature.
The limits on the electron neutrino and muon neutrino masses are set at a 90\% confidence level 
while the tau neturino limits are set at a 95\% confidence level.}
\end{table}

The quarks and leptons can each be divided up into three ``generations'' composed of pairs
of particles with identical charges but whose masses increase with each generation.
The generations are labeled in \tab\ref{tab:theory_fermions}. 
In the leptonic sector, leptons only interact exclusively with leptons of their own generation.
However, in the quark sector, while there is a strong preference for the quarks to interact
within the same generation, it is still possible for quarks to interact with quarks of 
the other generations. This is described by the CKM matrix... %elaborate
Even though there are three 
generations in both the lepton and quark sectors, 
the quarks and leptons are not observed to interact directly, 
thus the quark and lepton generations should be thought of as separate.


The SM can be written down using a lagrangian of the form
\begin{equation}
\curlyl_{\textrm{SM}} = \curlyl_{\textrm{QCD}} + \curlyl_{\textrm{EW}} + \curlyl_{\textrm{EWSB}}
\end{equation}
which is gauge invariant or something...
From this, one can calculate all of the fundamental interactions of the SM.
As written, the SM lagrangian can be split up into separate terms describing the
QCD, EW, and EWSB behavior. 
The EWSB term includes the behavior related to fermion mass generation.
The details of each term are described in more detail below.

CP violation?

\subsection{Quantum Chromodynamics}
The QCD term in the SM lagrangian can be written as
\subsection{Parton Distribution Functions}
\subsection{The Electroweak Theory}
The EW term in the SM lagrangian can be written as
\subsection{Electroweak Symmetry Breaking}
The EWSB term in the SM lagrangian can be written as



\subsection{The \dubya~Boson}
Of most interest to the topic of this thesis is the behavior and properties
of the \dubya~boson, the charged gauge boson of the EW theory.
The \dubya~was first discovered in 1983 via $p\overline{p}$ collisions
at SPS Synchroton by looking at its decay to an electron %abbreviation?
and electron neutrino (cite).
Its mass has been measured in $p\overline{p}$ collisions at the Tevatron
and in $e^{+}e^{-}$ collisions at LEP to give a world 
average of $80.385 \pm 0.015\GeV$~\cite{PDG:2014}.
More recently, measurements of the mass have also been 
performed in $pp$ collisions at the LHC have been shown to be consistent
with \_\_ reported from ATLAS (cite) and \_\_\_ from CMS (cite).
The width assuming a Breit-Wigner distribution has also been measured 
at LEP and the Tevatron with an average value 
of $2.085\pm0.042\GeV$ \cite{PDG:2014}. %what about the LHC ?

refer to isospin and hypercharge?
lepton universality?

The \dubya~decays into quarks roughly 2/3 of the time. 
The remaining third of the time the \dubya~decays approximately 
evenly into each of the three lepton generations.
The measured branching fractions are summarized in \tab\ref{tab:theory_wdecay}.
The leptonic decays of the \dubya~result in a charged lepton 
with the same charge as the parent \dubya~(as dictated by charge conservation)
and a neutrino (or anti-neutrino if the parent \dubya~has negative charge).
Thus, the allowed leptonic decays are of the form:
\begin{align*}
W^{-} &\rightarrow l^{-} \overline{\nu}_l \\
W^{+} &\rightarrow l^{+} \nu_l
\end{align*}

\begin{table}[ht]
\begin{tabular}{c}
blah 
\end{tabular}
\caption{Measured branching fractions of the \dubya~boson as reported
by the Particle Data Group \cite{PDG:2014}}
\label{tab:theory_wdecay}
\end{table}

maybe show W mass plot from PDG or distribution from ATLAS?

The CKM...


\section{Signal}
\subsection{WWW Signal}
\subsection{Effective Field Theory}
\subsubsection{Anomalous Quartic Gauge Couplings}
%be sure to add a review of measurements

A measurement
of the production rate can be used to probe the gauge couplings, in
particular, the process is sensitive to quartic gauge couplings. The
{\sc VBFNLO } code has implemented a list of higher order operators
that parameterize the effects of new physics at energy scale beyond
the reach of current collider experiments.  The effective field theory
approach is practical and widely used when there is no compelling
specific model of new physics beyond the SM, see for example,
discussions in Refs:~\cite{Hagiwara:1993ck}, \cite{Buchmuller:1985jz}
and \cite{Eboli:2006wa}.  As a benchmark, we choose two gauge
invariant dimension-8 operators:
\begin{equation}
\mathcal{L}_{s,0} = [(\mathrm{D}_\mu\phi)^\dag\mathrm{D}_\nu \phi]\times [(\mathrm{D}^\mu\phi)^\dag\mathrm{D}^\nu \phi]
\end{equation}
\begin{equation}
\mathcal{L}_{s,1} = [(\mathrm{D}_\mu\phi)^\dag\mathrm{D}^\mu \phi]\times [(\mathrm{D}_\nu\phi)^\dag\mathrm{D}^\nu \phi]
\end{equation}
where $\phi$ is the Higgs field doublet, and $\mathrm{D}_\mu$ is the covariant derivative. 
The Lagrangian of the effective field theory is thus: 
\begin{equation}
\mathcal{L}_{eff} = \mathcal{L}_{SM} + \frac{f_{s0}}{\Lambda^4}\mathcal{L}_{s,0}+\frac{f_{s1}}{\Lambda^4}\mathcal{L}_{s,1}
\end{equation}
The choice of the two operators is introduced to benchmark possible
deviations from the SM. If a significant excess of events is 
observed in the data, the parameterization will be changed, to incorporate 
more operators, to
investigate the nature of the observed new physics.
% Once statistically significant
% deviation is observed, of course, more operators should be considered
% for investigating the nature of the observed new physics.




