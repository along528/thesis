\chapter{Theory}
\label{sec:theory}

%Here I need to go through and describe all of the things that are important for the discussion
%aim for 20 pages

%things to discuss include:
%- A general introduction to the standard model (probably no more than a page or two)
%- Electroweak physics
%- W decay
%- parton distribution functions (maybe with some thorough discussion on this) as well as renormalization and factorization scales

%- WWW production (do I actually want to discuss this at all here??)
%- Effective field theory and anomalous coupling measurements


%assuming these will be defined here
% Quartic Gauge Couplings (QGC) 
% Standard Model (SM)
% anomalous Quartic Gauge Couplings (aQGC)
% Parton Distribution Function (PDF)


We can describe the probability of a given process to be produced by calculating its cross-section, measured in barns ($1~\textrm{barn}=10^{-24}~\textrm{cm}^2$).


Since a $W$ boson is too short lived to ever be measured directly, we
must instead study its decay products. 
The $W$ boson can decay into two different categories of 
particles: either
``leptonically'' into a charged lepton ($\ell$) and a 
neutrino ($\nu$) or ``hadronically''
into a pair of quarks.
Quarks are particles that interact with the Strong force and are 
usually measured as complicated
collections of particles called jets.
The charged leptons include the 
electrons, $e^{\pm}$, familiar because of their presence in stable matter,
and the less familiar muons, $\mu^{\pm}$, and taus, $\tau^{\pm}$, 
which themselves decay before forming stable matter.  
Muons live long enough to be measured
in the ATLAS detector while tau leptons do not. Because of this, we 
only refer to electrons and muons when discussing the charged leptons.
There are also three different types of neutrinos, named according to which
charged lepton they pair with 
during $W$ decay;  these are the electron neutrino ($\nu_e$),
muon neutrino ($\nu_{\mu}$), and tau neutrino ($\nu_{\tau}$). 
Neutrinos are peculiar 
because they are notoriously difficult to measure.  
As a result, we can only infer their presence after measuring all 
of the particles in the detector and taking into account
that the momentum is constrained at the collision point. If the momentum 
from all of the particles that were measured does not meet this 
constraint, the imbalance is attributed to the
presence of neutrinos. We call this ``missing energy''.  Thus, the 
leptonic $W$  decay channel is measured
experimentally by the presence of a charged lepton 
(electron or muon) plus missing energy.



\section{Standard Model}
\subsection{The Electroweak Theory}
\section{Effective Field Theories}

A measurement
of the production rate can be used to probe the gauge couplings, in
particular, the process is sensitive to quartic gauge couplings. The
{\sc VBFNLO } code has implemented a list of higher order operators
that parameterize the effects of new physics at energy scale beyond
the reach of current collider experiments.  The effective field theory
approach is practical and widely used when there is no compelling
specific model of new physics beyond the SM, see for example,
discussions in Refs:~\cite{Hagiwara:1993ck}, \cite{Buchmuller:1985jz}
and \cite{Eboli:2006wa}.  As a benchmark, we choose two gauge
invariant dimension-8 operators:
\begin{equation}
\mathcal{L}_{s,0} = [(\mathrm{D}_\mu\phi)^\dag\mathrm{D}_\nu \phi]\times [(\mathrm{D}^\mu\phi)^\dag\mathrm{D}^\nu \phi]
\end{equation}
\begin{equation}
\mathcal{L}_{s,1} = [(\mathrm{D}_\mu\phi)^\dag\mathrm{D}^\mu \phi]\times [(\mathrm{D}_\nu\phi)^\dag\mathrm{D}^\nu \phi]
\end{equation}
where $\phi$ is the Higgs field doublet, and $\mathrm{D}_\mu$ is the covariant derivative. 
The Lagrangian of the effective field theory is thus: 
\begin{equation}
\mathcal{L}_{eff} = \mathcal{L}_{SM} + \frac{f_{s0}}{\Lambda^4}\mathcal{L}_{s,0}+\frac{f_{s1}}{\Lambda^4}\mathcal{L}_{s,1}
\end{equation}
The choice of the two operators is introduced to benchmark possible
deviations from the Standard Model. If a significant excess of events is 
observed in the data, the parameterization will be changed, to incorporate 
more operators, to
investigate the nature of the observed new physics.
% Once statistically significant
% deviation is observed, of course, more operators should be considered
% for investigating the nature of the observed new physics.




