\section{Data and Simulation Samples}
\subsection{Data}
\label{sec:subsection_data}



This analysis is based on the study of the full proton-proton collision
data from the LHC in 2012. The amount 
of data used in this analysis corresponds to 
an integrated luminosity of \lumi.
The uncertainty on the integrated luminosity is $1.9\%$ 
following the same methodology as in \cite{Aad:2013ucp}.
%from Van-der-Meer scans taken throughout 2012.
%read this citation.
The data are selected after requiring that at least one
of a series of the single lepton triggers passed during data taking, 
specifically, one of the following:
either an electron trigger 
requiring at least one isolated
electron with $\pt>24$~\GeV~, an electron trigger requiring
at least one (possibly non-isolated) electron 
with $\pt>60$~\GeV, a muon 
trigger requiring at least one isolated muon with $\pt>24$~GeV,
or a muon trigger requiring at least one 
(possibly non-isolated) muon with $\pt>36$~GeV.


\subsection{Simulation samples}
%%Do I need to talk about Monte Carlo showering, 
%%hadronization and reconstruction?
%%A general discussion of Monte Carlo could go here

An important tool for the modeling of physics processes
that at the LHC is Monte Carlo simulation (MC).
MC relies on random sampling to connect the matrix element formulations
derived from quantum mechanical perturbation theory into 
actual predictions for the results of proton-proton collisions
at the LHC.
The prediction of a single collision from the MC represents
one possible outcome of the proton-proton collision, with all of the 
products of the hard-scattering and their four-momenta.
This result can be passed through additional MC simulation to describe
hadronization and the soft products of the collision e.g. photon radiation.
Finally, these products are passed through a detailed 
simulation of the response of the 
ATLAS detector built in \geant~\cite{Agostinelli:2002hh}
so that the same reconstruction algorithms
can be applied as in the data.
This sampling is repeated many times to populate the 
distribution of possible
outcomes. Dedicated MC programs are provided by theorists for 
different processes and to different orders in perturbation theory.
Details of the different processes simulated from MC and their
treatment are presented below.




\subsubsection{Signal Processes}
\label{sec:signal}


%%%
%If I'm going to include this, I better read up on it.
%%%
%The production \xsec without Higgs contribution has been calculated 
%to $\mathcal{O}(\alpha_s)$  corrections in Ref~\cite{Binoth:2008kt}.
%$\mathcal{O}(\alpha_s)$ corrections, Higgs boson exchange and spin 
%correlations of $W$ bosons lepton decay are also available
%~\cite{Campanario:2008yg}.  

The SM $WWW$ signal processes are implemented in the Monte
Carlo generator \vbfnlo~\cite{Arnold:2011wj,Arnold:2012xn},
which can generate partonic events at leading-order (LO) in QCD with
next-to-leading-order (NLO) cross-sections, 
and in \madgraph~\cite{MadGraph}, which can generate
partonic events at NLO  with NLO cross-sections. 
The partonic events are further processed 
by \pythiaeight~\cite{Sjostrand:2007gs} and \photos~\cite{Golonka:2005pn} 
to add effects of beam remnant interactions and initial and 
final state radiation. 
SM parameters must be provided to the MC generators as input. 
The most relevant input parameters are listed 
for the generators in \tab\ref{tab:signal_sm_parameters}.
The parameters are set in \pythiaeight~ using the ATLAS tune 
of AU2\cite{atlas:2011zja}.
The MC generators must also be provided an appropriate PDF.
The PDF used  in the LO \vbfnlo~generation is
the LO CTEQ6L1~\cite{Pumplin:2002vw} PDF set;
CT10 NLO~\cite{guzzi:2011sv}
is used in the NLO \vbfnlo~cross-section calculation.
The PDF used in the NLO \madgraph~generation 
and \xsec~calculation is CTEQ6L1 
but this is re-weighted to CT10 NLO using a k-factor.
Since the MC generators are computed to finite order in perturbation
theory, renormalization and factorization scales must be chosen.
The renormalization and factorization scales are dynamically
set to the $WWW$ invariant mass in the \vbfnlo~samples; they 
are set to a fixed scale equal to the $Z$ mass in \madgraph.
The \vbfnlo~samples are restricted to leptonic decays of the $W$~bosons
where each lepton has a \pt~of at least 5~\GeV. The \madgraph~
samples include all decays of the $W$~boson, with a requirement 
that jets have a a \pt~of at least 10~\GeV~ but with no requirement
on the \pt~of leptons.
The \vbfnlo~ and \madgraph~samples handle interference 
between $WH\rightarrow WWW(*)$ 
and on-shell $WWW$ production at LO, but \madgraph~is not
able to do this at NLO. As a result, the NLO \madgraph~samples
are split by on-shell \www~ and $WH\rightarrow WWW(*)$ production.
Both sets are further split by the \www~charge mode.
For each sample, the \xsecs are summarized in \tab\ref{tab:signal_xsec} 
in their full phase space and in a common fiducial phase space
defined in \sec\ref{sec:fiducial}.  
The fiducial \xsecs are observed to be nearly the same
between the two generators.
This serves as a good check of the understanding of the 
signal process. The \madgraph~\xsecs are used throughout the 
remainder of the analysis.

%Do I need to describe the k-factors?
%It would be nice to also add some distributions from Rivet comparing
%the two at truth level.


%Describe the pdf uncertainty calculation.
%what about renormalization and factorization scales




\begin{table}[ht]
\centering
\begin{tabular}{|l||c|c||}
\hline
 & \vbfnlo & \madgraph \\
\hline
\hline
Higgs mass, $m_H$ & 126.0~\GeV & \\ 
Top mass, $m_t$ & 172.4~\GeV  & \\
$Z$ mass, $m_Z$ & 91.1876~\GeV & 91.188~\GeV\\
$W$ mass, $m_W$ & 80.398~\GeV & \\
Fermi constant, $G_F$ & $1.16637\times 10^{-5}~\GeV^{-2}$ & \\
\hline
\end{tabular} 


\caption{List of the most relevant SM parameters used as input to the 
signal MC generation.}
\label{tab:signal_sm_parameters}
\end{table}


\begin{table}[ht]
\centering
\begin{tabular}{|cc||c|c|c|}
\hline
\multicolumn{2}{|c||} {Sample} &  \multicolumn{2}{c|}{Cross-section [fb]} \\
                              && Inclusive & Fiducial \\
\hline
\hline
%\multirow{3}{*}{\vbfnlo~LO} & $W^{+}W^{+}W^{-}\rightarrow l\nu l\nu l\nu$ & $3.56 \pm 0.005$ & \\
%                            & $W^{-}W^{+}W^{-}\rightarrow l\nu l\nu l\nu$ & $1.88\pm0.003$ & \\ 
%			    \cline{2-4} 
%                            & Sum & $5.44\pm0.006$ & \\ 
%\hline
\multirow{3}{*}{\vbfnlo~NLO} & $W^{+}W^{+}W^{-}\rightarrow l\nu l\nu l\nu$ & $4.95 \pm 0.007$ & $0.2050 \pm 0.0070$\\
                           & $W^{-}W^{+}W^{-}\rightarrow l\nu l\nu l\nu$ & $2.65\pm0.004$ & $0.0987 \pm 0.0037$\\ 
			    \cline{2-4} 
                           %& $WWW\rightarrow l\nu l\nu l\nu$ & $7.60\pm0.008$ & \\ 
                           & Sum & $7.60\pm0.008$ & $0.3037 \pm 0.0072$\\ 
\hline
%With PDF KFactor
\multirow{5}{*}{\madgraph~NLO} & $W^{+}W^{-}W^{+}\rightarrow \textrm{Anything}$ &$59.47\pm0.11$ & $0.0900 \pm 0.0048$\\
                        & $W^{-}W^{+}W^{-} \rightarrow \textrm{Anything}$& $28.069\pm0.076$ & $0.0476 \pm 0.0043$\\
                        & $W^{+}H\rightarrow W^{+}W^{+}W^{-}(*)\rightarrow\textrm{Anything}$ & $99.106\pm0.019$ & $0.1114 \pm 0.0029$\\
                        & $W^{-}H\rightarrow W^{-}W^{+}W^{-}(*) \rightarrow \textrm{Anything}$& $54.804\pm0.010$ & $0.0603 \pm 0.0015$\\
			\cline{2-4} 
                        & Sum & $241.47\pm0.13$ & $0.3092 \pm 0.0072$\\
%Without PDF KFactor
%\multirow{5}{*}{\madgraph~NLO} & $W^{+}W^{-}W^{+}\rightarrow \textrm{Anything}$ &$55.07\pm0.10$ & $0.0818 \pm 0.0044$\\
%                        & $W^{-}W^{+}W^{-} \rightarrow \textrm{Anything}$& $25.99\pm0.07$ & $0.0433 \pm 0.0039$\\
%                        & $W^{+}H\rightarrow W^{+}W^{+}W^{-}(*)\rightarrow\textrm{Anything}$ & $91.765\pm0.018$ & $0.1013 \pm 0.0026$\\
%                        & $W^{-}H\rightarrow W^{-}W^{+}W^{-}(*) \rightarrow \textrm{Anything}$& $50.7440\pm0.0094$ & $0.0548 \pm 0.0014$\\
%			\cline{2-4} 
%                        & Sum & $223.57\pm0.12$ & $0.2812 \pm 0.0066$\\
\hline
\end{tabular}

\caption{Inclusive and common fiducial cross-sections at NLO 
for \vbfnlo~and \madgraph~samples. 
The sum of the inclusive \xsecs are different
because of the different branching fractions in the two cases. 
The sum of the fiducial cross-sections, however, are expected to be similar because
they are computed for the same phase space, as described in \sec...}
\label{tab:signal_xsec}
\end{table}


%%%%
%soud I show the dependence on scales here?
%%%%
%The dependencies of the 
%$\xsecs on the choices of scales have been studied in the two
%references~\cite{Binoth:2008kt,Campanario:2008yg}. 

% The production at LO is a pure electroweak process. The
% NLO correction brings in $\alpha_s$ which actually makes the cross
% sections more sensitive to the choices of scales. 
%It has been pointed
%out that a jet veto should reduce the scale dependence. 

%The $W$ boson is short lived, so one must study its decay products.
%As already mentioned, the focus of this analysis is on the final state 


%need to also show the MadGraph parameters
%maybe rephrase so that I can discuss both in parallel
%get generation parameters from semi-leptonic note
%report both sets of cross-sections here
%include updated info on cross-sections and PDFs 

\begin{figure}[ht!]
\centering
\includegraphics[width=.35\columnwidth]{figures/pdf/MADppm_total_cteq6l1.png}
\includegraphics[width=.35\columnwidth]{figures/pdf/MADpmm_total_cteq6l1.png}
\includegraphics[width=.35\columnwidth]{figures/pdf/MADppm_fiducial_cteq6l1.png}
\includegraphics[width=.35\columnwidth]{figures/pdf/MADpmm_fiducial_cteq6l1.png}
\caption{The signal cross-sections for different PDFs along with their
uncertainties are shown on the {\sc MadGraph} $WWW$ signal samples
for the total $WWW$ phase space and branching fraction for
the $W^{+}W^{+}W^{-}$ (top left) and $W^{+}W^{-}W^{-}$ (top right)
charge modes
and in the fiducial region for $W^{+}W^{+}W^{-}$ (bottom left) 
and $W^{+}W^{-}W^{-}$ (bottom right).
The bands show the PDF uncertainty for CT10 NLO (solid yellow),
MSTW 2008 NLO (hashed blue), and NNPDF 3.0 NLO (hashed red). The
solid line shows the envelope of all uncertainty bands used as the final
PDF uncertainty estimate. The central value of CT10 NLO is taken as the
central value of the estimate.
The dashed-line shows the cross-section and 
statistical uncertainty for the CTEQ6L1
pdf sets used in the original generation step.}
\label{fig:signal_pdf_unc}
\end{figure}

\begin{table}[ht!]
\centering
\begin{tabular}{c|c|c}
\hline
 & \multicolumn{2}{c}{PDF Uncertainty}\\
 & $W^{+}W^{+}W^{-}$ & $W^{+}W^{-}W^{-}$ \\
\hline
\hline
Total & $+2.58\%~-2.51\%$ &  $+8.69\%~-3.47\%$ \\
Fiducial & $+3.64\%~-3.00\%$ & $+7.57\%~-3.08\%$ \\
\hline
\end{tabular}
\caption{Summary of PDF uncertainties estimated on NLO {\sc MadGraph} cross-sections
in both the fiducial and total phase space.}
\label{tab:pdfunc}
\end{table}

The uncertainty due to the choice of PDF is derived for the {\sc MadGraph} 
cross-sections following a modified version of the pdf4lhc
\cite{Botje:2011sn} recommendations.  The resulting 
uncertainty is shown separately for the two different charge modes
in both the fiducial and the inclusive phase
space in Table~\ref{tab:pdfunc}.
The uncertainty is determined by comparing three different PDFs:
CT10 NLO~\cite{Lai:2010vv}, MSTW2008 NLO~\cite{Martin:2009iq}, 
and NNPDF 3.0 NLO~\cite{Ball:2014uwa}. 
This comparison is presented in Figure~\ref{fig:signal_pdf_unc}.  
Symmetric 68\% CL uncertainties 
are determined for CT10 NLO and MSTW 2008 NLO using the 68\% CL 
set provided for MSTW directly and the 90\%CL set for CT10 after
scaling down by 
a factor of 1.645 in order to approximate a 68 \% CL uncertainty. 
The uncertainty of the NNPDF 3.0 NLO PDF set is 
determined by using the standard deviation of the distribution 
of 101 MC PDFs provided in the PDF set; the nominal value is taken
from the mean of the same PDFs.  
The CT10 NLO PDF central value is used as the nominal 
value of the final estimate.
The final PDF uncertainty on that estimate is
taken as the envelope of the uncertainty bands for all three PDF sets.  



The uncertainty on the factorization and renormalization scales are 
determined by varying each of them independently up or down by 
a factor of two. 
The effect of these variations on the cross-sections
as compared to the nominal
are shown separately for the two different charge 
modes in \tab~\ref{tab:scaleVariation}.
The symmetric uncertainty is then determined by taking the maximum 
variation for each charge mode, 
namely, 2.62\% for $W^+W^+W^-$ and 2.53\% for $W^-W^+W^-$. 

\begin{table}[ht!]
    \centering
\begin{tabular}{cc|ccc}
\hline
& \backslashbox{$\mu_F$}{$\mu_R$}     & $\frac{1}{2}M_{WWW}$ & $M_{WWW}$ &  $2M_{WWW}$ \\
\cline{2-5}
\multirow{3}{*}{\Wp\Wp\Wm} &$\frac{1}{2}M_{WWW}$ & 2.62\% & -0.14\% & -2.11\% \\
%\cline{2-5}
&$M_{WWW}$ & 2.13\% & 0 & -2.41\% \\
%\cline{2-5}
&$2M_{WWW}$ & 1.56\% & 0.24\% & -2.42\% \\
\hline
\hline
& \backslashbox{$\mu_F$}{$\mu_R$}     & $\frac{1}{2}M_{WWW}$ & $M_{WWW}$ &  $2M_{WWW}$ \\
\cline{2-5}
\multirow{3}{*}{\Wm\Wp\Wm} &$\frac{1}{2}M_{WWW}$ & 1.91\% & 1.38\% & -2.00\% \\
%\cline{2-5}
&$M_{WWW}$ & 1.61\% & 0 & -2.53\% \\
%\cline{2-5}
&$2M_{WWW}$ & 1.25\% & -1.05\% & -2.12\% \\
\hline
\end{tabular}
\caption{The relative variation of the NLO cross sections corresponding 
to different choices of factorization and renormalization 
scales for the \Wp\Wp\Wm and \Wm\Wp\Wm  processes. }
\label{tab:scaleVariation}
\end{table}

The signal cross-sections and uncertainties are thus determined to be 
\begin{equation}
\sigma^{\textrm{Total}}_{\textrm{Theory}}= 241.47\pm0.13 ~(\textrm{Stat.}) ~^{+10.33}_{-6.08} ~(\textrm{PDF}) ~\pm 6.3 ~(\textrm{Scale}) ~\textrm{fb} %uncertainty?
\end{equation}
for the inclusive \xsec and
\begin{equation}
\label{eq:fiducial_theory}
\sigma^{\textrm{Fiducial}}_{\textrm{Theory}}= 309.2\pm7.2 ~(\textrm{Stat.}) ~^{+15.05}_{-8.36} ~(\textrm{PDF}) ~\pm 8.0 ~(\textrm{Scale}) ~\textrm{ab} %uncertainty?
\end{equation}
for the fiducial cross-section.


%should i include this
%The analysis considers events with three leptons ($e$ or $\mu$) in the final state. The contributions from events in which $W$ bosons decay to $\tau$'s, and the $\tau$'s sequentially decay to $e$ or $\mu$ should be included and is expected to be 40\% of total yield of the 3-lepton final state.  


\subsubsection{aQGC signal}
blank
%
MC samples of the aQGC signal processes described in \sec\ref{sec:eft}
have been generated using \vbfnlo at NLO in QCD.  (but don't we use LO?)
The cross-sections for the aQGC signal depend on the values
of the couplings $f_{s,0}$ and $f_{s,1}$. MC samples have 
been generated for a grid of points in the $f_{s,0}$ vs $f_{s,1}$ space
and their cross-sections are shown in \fig\ref{fig:aqgc_total_xsec_ununitarized_3l}. %histogram of cross-sections

\begin{figure}[ht!]
\centering
\includegraphics[width=.8\textwidth]{figures/aQGC/total_xsec/www_3l_aqgc_total_ununitarized_noratio.png}
\caption{Total cross-section for non-unitarized aQGC signal samples as a function of $f_{s,0}$ vs $f_{s,1}$.
The total SM cross-section is shown at $f_{s,0}=f_{s,1}=0$ for comparison.}
\label{fig:aqgc_total_xsec_ununitarized_3l}
\end{figure}

The issues of unitarity violation \sec\ref{sec:eft} are taken
into account using a form factor like in \eqn\eqref{eq:form_factor}.
The choices of the exponent, $n$, and form factor scale, $\Lambda$, 
are somewhat ad-hoc. Furthermore, a complete study of the unitarity
behavior of this process has never been performed, so there are not
currently detailed prescriptions on what to choose. 
However, based on discussions with the authors of \vbfnlo, who
are at the moment trying to perform these studies, an exponent
of $n=1$ is expected to be sufficient to achieve unitarity 
for this process.  As for the choice of $\Lambda$, we have
chosen to look at a few different values, which cover a wide
range but which should follow a smooth interpolation. 
This has the advantage of providing information about the
sensitivity to the form factor that can be interpreted 
by theorists as they see fit. Dedicated MC samples
are generated with the unitarization applied for values
of $\Lambda =$ 500~\GeV, 1000~\GeV, 2000~\GeV, and 3000~\GeV.
The cross-sections for each of these unitarization cases
are shown in \fig\ref{fig:aqgc_total_xsec_unitarized_3l}.

\begin{figure}[ht!]
\centering
\includegraphics[width=.45\textwidth]{figures/aQGC/total_xsec/www_3l_aqgc_total_3TeV_noratio.png}
\includegraphics[width=.45\textwidth]{figures/aQGC/total_xsec/www_3l_aqgc_total_2TeV_noratio.png}
\includegraphics[width=.45\textwidth]{figures/aQGC/total_xsec/www_3l_aqgc_total_1TeV_noratio.png}
\includegraphics[width=.45\textwidth]{figures/aQGC/total_xsec/www_3l_aqgc_total_p5TeV_noratio.png}
\caption{Total cross-section for unitarized aQGC signal samples as a function of $f_{s,0}$ vs $f_{s,1}$.
Four different values of the unitarization scale, $\Lambda$, are chosen: 3~\TeV~(Top Left),
2~\TeV~(Top Right), 1~\TeV~(Bottom Left), and 0.5~\TeV~(Bottom Right).
The total SM cross-section is shown at $f_{s,0}=f_{s,1}=0$ for comparison.}
\label{fig:aqgc_total_xsec_unitarized_3l}
\end{figure}
























\newpage
\subsubsection{Background samples}
\label{sec:www_bg_samples}

There are other processes produced in proton-proton collisions at the LHC
which can mimic the signal processes. These are referred to as background processes.
In many cases, the background processes are either
more abundant than or of a similar abundance to
the signal. As a result, they must be well understood if there is any hope
of distinguishing between the two. The background processes to the signal
fall into two general categories: irreducible and reducible. 
The irreducible backgrounds are those that have the exact same final
state as the signal. Thus, they 
are characterized by having either exactly three prompt leptons, meaning they
come directly from the hard scattering process.
The reducible backgrounds are those which do not have the exact same
final state as the signal, but can mimic the signal in some circumstances.
For our signal, this includes backgrounds with four or more prompt leptons,
where only three leptons are measured;
two prompt leptons and an isolated photon, which can mimic an electron,
referred to as the photon backgrounds;
or two prompt leptons and a jet that mimics a lepton, referred to as
the fake backgrounds.
We treat similarly those backgrounds with three or more prompt leptons,
hereby referred to as the prompt background.
The prompt and photon backgrounds 
are estimated primarily using MC simulation while the fake background
is estimated using the data itself. 
This will be described in more detail in \sec\ref{sec:bg_fake}.
For now, we will focus only on the processes estimated using MC simulation.

Of the prompt backgrounds,
the $WZ$ process is the most important contribution since it has a 
large cross-section (compared to the signal)
and results in a final state with exactly three leptons. Another important 
prompt background is the $ZZ$ process,
which has a similar cross-section to the $WZ$ process, but is typically 
selected when 
four leptons are produced but one escapes detection.
Thus, this process is suppressed by the 
efficiency for not measuring the presence of a lepton. 
These are collectively referred to as the di-boson processes, sometimes
indicated as $VV$ where $V = W$ or $Z$\footnote{The $WW$ process is also considered
but can only produce at most two prompt leptons, making it negligible.}. 
The di-boson processes are produced using the 
\powheg~\cite{Alioli:2008gx,Nason:2004rx,Frixione:2007vw,Alioli:2010xd} generator
with the CT10 NLO PDF set and 
hadronized through \pythiaeight~using the AU2 tune, same as the signal.
Other prompt backgrounds 
include tri-boson processes like $ZWW$ and $ZZZ$ 
(referred to collectively as $VVV$)
and \ttV~production. Tri-boson processes
have cross-sections of a similar size to the signal but are suppressed 
for a similar reason
as the $ZZ$, since these can produce either four or six lepton final 
states. 
The \ttV~production process occurs when a vector
boson is produced in association with a \tt~pair. 
Since the top quark almost always decays
into a $W$-boson and a $b$-quark, \ttV~production also results in 
three vector bosons which decay into a three or four lepton
final state.
The $VVV$ and \ttV~processes were generated using \madgraph~with the 
CTEQ6L1 PDF set and hadronized
using \pythiasix~\cite{PYTHIA} with the AUET2B~\cite{atlas:2011zja} 
tune.

The photon backgrounds occur entirely from the di-boson process $Z\gamma$
where the $Z$ boson decays to two leptons and the photon mimics an electron.
A photon is measured
by observing an energy deposit in the electromagnetic calorimeter 
without any associated track in the inner detector.
A photon can mimic an electron
if it converts into an electron-positron
pair while still inside the inner detector. This leaves a track 
in the inner detector plus an energy deposit in the 
calorimeter, which is the tell-tale sign of an electron.
The $Z\gamma$ samples were generated with the \sherpa~\cite{sherpa} generator 
and the CT10 PDF set.  %hadronization? CT10 NLO? Or LO?
In addition to this process, the $W\gamma$ process behaves similarly 
but only has one prompt lepton in addition to the photon, so it is negligible.
Still, we generate it by using
the \alpgen~\cite{ALPGEN} generator with the CTEQ6L1 PDF set
and hadronize it using \jimmy~\cite{Jimmy} with the AUET2C~\cite{atlas:2011zja} 
tune.

Some of the di-boson and tri-boson processes just discussed can also be produced
through loop induced processes or double parton scattering (DPS).
The $WW$ and $ZZ$
loop induced processes are generated using the gg2ZZ~\cite{Binoth:2008pr} 
and gg2WW~\cite{Binoth:2006mf} generators with the CT10 PDF set and
hadronized using JIMMY with the AU2 tunes.
The DPS
processes are generated using \pythiaeight~with the AU2 
tunes and the CTEQ6L1 PDF set. 

The fake background is nominally estimated using the data
as described in \sec\ref{sec:bg_fake}. Some of the contributions
to this background, however, can be simulated using MC 
for cross-checks of 
the estimate from data. The main contributions
to the fake background
are the single boson processes ($V+$jets) and \tt~production.
The probability for a jet to mimic a lepton is actually quite small
and thus difficult to capture with adequate statistics using MC. 
However, these processes also have very large cross-sections.
Combining the two means that in fact the occurence of a jet mimicking
a lepton is not rare and thus non-negligible. 
The single boson $Z+$jets processes are generated using \sherpa~with the CT10
PDF set; the $W+$jets processes are generated using \alpgen~with
the CTEQ6L1 PDF set and hadronized using \jimmy~with the AUET2C tunes.
For the $Z+$jets samples, special care must be taken to remove any overlap 
between with the $Z\gamma$ simulated samples described earlier.
The \tt~processes are generated using the \mcatnlo~\cite{MCatNLO}
generator with the CT10 PDF set and hadronized in JIMMY.  %what is the tune?
Finally, the fake background also has contributions from single top production,
though it is less important. Single top production is simulated separately 
for three different production mechanisms, differing in their initial
and final states, known as
s-channel ($qb\to q't$), t-channel ($q'\overline{q}\to \overline{b}t$), 
and $Wt$-channel ($bg \to Wt$). The s-channel 
and $Wt$-channel are generated using \mcatnlo~with the CT10 PDF set and 
hadronized through \jimmy~; the t-channel is generated using 
\madgraph~with the CTEQ6L1 PDF set and hadronized 
using \pythiasix~with the AUET2B tunes.





