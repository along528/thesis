\subsubsection{Background samples}
\label{sec:www_bg_samples}

There are other processes produced in proton-proton collisions at the LHC
which can mimic the signal processes. These are referred to as background processes.
In many cases, the background processes are either
more abundant than or of a similar abundance to
the signal. As a result, they must be well understood if there is any hope
of distinguishing between the two. The background processes to the signal
are characterized by having either at least three prompt leptons, meaning they
come directly from the hard scattering process;  
two prompt leptons and an isolated photon, which can mimic an electron;
or two prompt leptons and a jet that mimics a lepton.
The first two are estimated primarily using MC simulation; the third type
is estimated using the data itself. 
This will be described in more detail in \sec\ref{sec:bg_fake}.
For now, we will focus only on the processes estimated using MC simulation.

The most important backgrounds are those with at least three prompt leptons, 
hereby referred to as the prompt backgrounds. Of these prompt backgrounds,
the $WZ$ process is the most important since it has a 
large cross-section (compared to the signal)
and results in a final state with exactly three leptons. Another important 
prompt background is the $ZZ$ process,
which has a similar cross-section to the $WZ$ process, but is typically 
selected by producing
four leptons and then not measuring one. Thus, this process is suppressed by the 
efficiency for not measuring the presence of a lepton. 
These are collectively referred to as the di-boson processes, sometimes
indicated as $VV$ where $V=W/Z$ (the $WW$ process is also considered
but can only produce at most two prompt leptons making it negligible). 
The di-boson processes are produced using the 
the \powheg~\cite{Alioli:2008gx,Nason:2004rx,Frixione:2007vw,Alioli:2010xd} generator
with the CT10 NLO PDF set and 
hadronized through \pythiaeight~using the AU2 tune, same as the signal.

Other prompt backgrounds 
include tri-boson processes like $ZWW$ and $ZZZ$ 
(typically referred to collectively as $VVV$)
and \ttV~production. Tri-boson processes
have cross-sections of a similar size to the signal but are suppressed 
for a similar reason
as the $ZZ$, since these can produce either four or six lepton final 
states. 
\ttV~production is when a vector
boson is produced in conjunction with a \tt~pair. 
Since the top quark almost always decays
into a $W$-boson and a $b$-quark, \ttV~production also results in an intermediate
state of three vector bosons which ultimately results in a three to four lepton
final state.
The $VVV$ and \ttV~processes were generated using \madgraph~with the 
CTEQ6L1 PDF set and hadronized
using \pythiasix~\cite{PYTHIA} with the AUET2B~\cite{atlas:2011zja} 
tune.

The second category of backgrounds to consider are those with two 
prompt leptons and a photon. We will call these the photon backgrounds.
The photon backgrounds occur entirely from the di-boson process $Z\gamma$
where the $Z$ boson decays to two leptons and the photon mimics an electron.
A photon is measured
by observing an energy deposit in the electromagnetic calorimeter 
without any associated track in the inner detector.
A photon can mimic an electron
if it converts into an electron-positron
pair while still inside the inner detector, thereby leaving a track 
in the inner detector while still leaving an energy deposit in the 
calorimeter, the tell-tale sign of an electron.
The $Z\gamma$ samples were generated with the \sherpa~\cite{sherpa} generator 
and the CT10 PDF set.  %hadronization? CT10 NLO? Or LO?
In addition to this process, the $W\gamma$ process behaves similarly 
but only has one prompt lepton in addition to the photon, so it is negligible.
Still, we generate it by using
the \alpgen~\cite{ALPGEN} generator with the CTEQ6L1 PDF set
and hadronize it using \jimmy~\cite{Jimmy} with the AUET2C~\cite{atlas:2011zja} 
tune.

Some of the di-boson and tri-boson processes just discussed can also be produced
through loop induced processes or double parton scattering (DPS).
The $WW$ and $ZZ$
loop induced processes are generated using the gg2ZZ~\cite{Binoth:2008pr} 
and gg2WW~\cite{Binoth:2006mf} generators with the CT10 PDF set and
hadronized using JIMMY with the AU2 tunes.
The DPS
processes are generated using \pythiaeight~with the AU2 
tunes and the CTEQ6L1 PDF set. 

The last category of backgrounds are those with prompt leptons plus
jets that mimic leptons, hereby 
referred to collectively as the fake background. 
The fake background is nominally estimated using the data
as described in \sec\ref{sec:bg_fake}. Some of the contributions
to this background, however, can be simulated using MC 
for cross-checks of 
the estimate from data. The main contributions
to the fake background
are the single boson processes ($V+$jets) and \tt~production.
These are processes with very large cross-sections so
that even though the probability for a jet mimicking a lepton is small,
the size of the cross-section means that their contribution is non-negligible.
The single boson $Z+$jets processes are generated using \sherpa~with the CT10
PDF set; the $W+$jets processes are generated using \alpgen~with
the CTEQ6L1 PDF set and hadronized using \jimmy~with the AUET2C tunes.
For the $Z+$jets samples, special care must be taken to remove any overlap 
between with the $Z\gamma$ simulated samples described earlier.
The \tt~processes are generated using the \mcatnlo~\cite{MCatNLO}
generator with the CT10 PDF set and hadronized in JIMMY.  %what is the tune?
Finally, the fake background also has contributions from single top production,
though it is less important. Single top production is simulated separately 
for the s-channel, t-channel, and $Wt$-channel. The s-channel 
and $Wt$-channel are generated using \mcatnlo~with the CT10 PDF set and 
hadronized through \jimmy~; the t-channel is generated using 
\madgraph~with the CTEQ6L1 PDF set and hadronized 
using \pythiasix~with the AUET2B tunes.

