\subsubsection{Background samples}
\label{sec:www_bg_samples}

There are other processes produced in proton-proton collisions at the LHC
which can mimic the signal processes. These are referred to as background processes.
In many cases, the background processes are either
more abundant than or of a similar abundance to
the signal. As a result, they must be well understood if there is any hope
of distinguishing between the two. The background processes to the signal
fall into two general categories: irreducible and reducible. 
The irreducible backgrounds are those that have the exact same final
state as the signal. Thus, they 
are characterized by having either exactly three prompt leptons, meaning they
come directly from the hard scattering process.
The reducible backgrounds are those which do not have the exact same
final state as the signal, but can mimic the signal in some circumstances.
For our signal, this includes backgrounds with four or more prompt leptons,
where only three leptons are measured;
two prompt leptons and an isolated photon, which can mimic an electron,
referred to as the photon backgrounds;
or two prompt leptons and a jet that mimics a lepton, referred to as
the fake backgrounds.
We treat similarly those backgrounds with three or more prompt leptons,
hereby referred to as the prompt background.
The prompt and photon backgrounds 
are estimated primarily using MC simulation while the fake background
is estimated using the data itself. 
This will be described in more detail in \sec\ref{sec:bg_fake}.
For now, we will focus only on the processes estimated using MC simulation.

Of the prompt backgrounds,
the $WZ$ process is the most important contribution since it has a 
large cross-section (compared to the signal)
and results in a final state with exactly three leptons. Another important 
prompt background is the $ZZ$ process,
which has a similar cross-section to the $WZ$ process, but is typically 
selected when 
four leptons are produced but one escapes detection.
Thus, this process is suppressed by the 
efficiency for not measuring the presence of a lepton. 
These are collectively referred to as the di-boson processes, sometimes
indicated as $VV$ where $V = W$ or $Z$\footnote{The $WW$ process is also considered
but can only produce at most two prompt leptons, making it negligible.}. 
The di-boson processes are produced using the 
\powheg~\cite{Alioli:2008gx,Nason:2004rx,Frixione:2007vw,Alioli:2010xd} generator
with the CT10 NLO PDF set and 
hadronized through \pythiaeight~using the AU2 tune, same as the signal.
Other prompt backgrounds 
include tri-boson processes like $ZWW$ and $ZZZ$ 
(referred to collectively as $VVV$)
and \ttV~production. Tri-boson processes
have cross-sections of a similar size to the signal but are suppressed 
for a similar reason
as the $ZZ$, since these can produce either four or six lepton final 
states. 
The \ttV~production process occurs when a vector
boson is produced in association with a \tt~pair. 
Since the top quark almost always decays
into a $W$-boson and a $b$-quark, \ttV~production also results in 
three vector bosons which decay into a three or four lepton
final state.
The $VVV$ and \ttV~processes were generated using \madgraph~with the 
CTEQ6L1 PDF set and hadronized
using \pythiasix~\cite{PYTHIA} with the AUET2B~\cite{atlas:2011zja} 
tune.

The photon backgrounds occur entirely from the di-boson process $Z\gamma$
where the $Z$ boson decays to two leptons and the photon mimics an electron.
A photon is measured
by observing an energy deposit in the electromagnetic calorimeter 
without any associated track in the inner detector.
A photon can mimic an electron
if it converts into an electron-positron
pair while still inside the inner detector. This leaves a track 
in the inner detector plus an energy deposit in the 
calorimeter, which is the tell-tale sign of an electron.
The $Z\gamma$ samples were generated with the \sherpa~\cite{sherpa} generator 
and the CT10 PDF set.  %hadronization? CT10 NLO? Or LO?
In addition to this process, the $W\gamma$ process behaves similarly 
but only has one prompt lepton in addition to the photon, so it is negligible.
Still, we generate it by using
the \alpgen~\cite{ALPGEN} generator with the CTEQ6L1 PDF set
and hadronize it using \jimmy~\cite{Jimmy} with the AUET2C~\cite{atlas:2011zja} 
tune.

Some of the di-boson and tri-boson processes just discussed can also be produced
through loop induced processes or double parton scattering (DPS).
The $WW$ and $ZZ$
loop induced processes are generated using the gg2ZZ~\cite{Binoth:2008pr} 
and gg2WW~\cite{Binoth:2006mf} generators with the CT10 PDF set and
hadronized using JIMMY with the AU2 tunes.
The DPS
processes are generated using \pythiaeight~with the AU2 
tunes and the CTEQ6L1 PDF set. 

The fake background is nominally estimated using the data
as described in \sec\ref{sec:bg_fake}. Some of the contributions
to this background, however, can be simulated using MC 
for cross-checks of 
the estimate from data. The main contributions
to the fake background
are the single boson processes ($V+$jets) and \tt~production.
The probability for a jet to mimic a lepton is actually quite small
and thus difficult to capture with adequate statistics using MC. 
However, these processes also have very large cross-sections.
Combining the two means that in fact the occurrence of a jet mimicking
a lepton is not rare and thus non-negligible. 
The single boson $Z+$jets processes are generated using \sherpa~with the CT10
PDF set; the $W+$jets processes are generated using \alpgen~with
the CTEQ6L1 PDF set and hadronized using \jimmy~with the AUET2C tunes.
For the $Z+$jets samples, special care must be taken to remove any overlap 
between with the $Z\gamma$ simulated samples described earlier.
The \tt~processes are generated using the \mcatnlo~\cite{MCatNLO}
generator with the CT10 PDF set and hadronized in JIMMY.  %what is the tune?
Finally, the fake background also has contributions from single top production,
though it is less important. Single top production is simulated separately 
for three different production mechanisms, differing in their initial
and final states, known as
s-channel ($qb\to q't$), t-channel ($q'\overline{q}\to \overline{b}t$), 
and $Wt$-channel ($bg \to Wt$). The s-channel 
and $Wt$-channel are generated using \mcatnlo~with the CT10 PDF set and 
hadronized through \jimmy~; the t-channel is generated using 
\madgraph~with the CTEQ6L1 PDF set and hadronized 
using \pythiasix~with the AUET2B tunes.

