

%No more than a few pages
%Include:
%- General features of the LHC and the injector chain
%- Luminosity and integrated luminosity
%- could discuss the details of the accelerator and machine parameters


%define these abbreviations here:
%LHC - Large Hadron Collider
The Large Hadron Collider (LHC) \cite{lhc} is
a 27 km circumference collider ring
located at CERN approximately 100 m underground on the 
of French-Swiss border near Geneva, Switzerland.
Its primary goal is to collide protons at very high energies (
the TeV scale)  similar to energies just a few trillionths of a second (?)
after the big bang.  %
The products of these collisions can 
be observed
by several independent but complementary detectors placed at different
points around the ring in order to probe the mechanism
for EWSB as well as to test and probe for new physics beyond the SM.
Since the dynamics of the collisions are governed by quantum mechanical
processes, the types of collision processes of interest cannot be
produced on demand, but instead occur at random with some
probability.
The probability for these processes is typically very small; thus, they are quite
rare (quote a number).
These processes are also short lived, meaning they do not live long
enough to reach the detector and are instead observed indirectly through their
decays. Since multiple physics processes can have the same decay 
signature, it is not possbile to say with certainty that a given 
collision comes from a specific physics process. Instead, 
we must count the number of observed collisions for a given signature
and compare this to the number expected from the 
quantum mechanical probabilities.  If the observed number differs
from the expected, then it could simply suggest that the theoretical expectation
is not well understood or it could suggest the presence of some new physics
process.
In order to make an adequate statement, we must be able to count
enough collisions of the desired signature (say 10 to 1000 depending on the
signature and its backgrounds) 
such that the statistical uncertainty is low.  This places a demand
on the LHC to produce as many collisions as possible, even of these rare
processes. To accomplish this, the LHC is thus designed to collide
protons at a maximum frequency of 40 MHz, or 40 million times per second!
More details about the LHC and collider physics in general are presented
below.



\section{Collider Physics}

talk about the principles

Bunches 

head on vs fixed target (grows linearly vs sqrt)



Motivation for proton-proton over p-pbar or ee
burn rate would be too high. (see slides C)

superconducting dipoles. must deal with quenches. trained.
limit on colliders

quadropoles and things for focusing

multiple stages of acceleration

Accelerate using RF. surfing

Must avoid synchrotron radiation


bunches

emittance?

collisions
\begin{equation}
N= L\sigma
\end{equation}

Units of $L$ are \lumiunits
For fixed target
\begin{equation}
L = N \rho_n t
\end{equation}

$N$ is the incident rate, $\rho_n$ is the target number
density, $t$ is the target thickness.

For colliding beam
\begin{equation}
L = \frac{N_1 N_2}{A} r_b
\end{equation}
where $N_1$ and $N_2$ are the number of protons per bunch, 
$A$ is the cross-sectional are of the beam, 
and $r_b$ is the crossing rate
\begin{equation}
r_b = n \frac{c}{C}
\end{equation}
$n$ is the number of bunches, $C$ is the circumference of the machine
and $c$ is ...

Or 
\begin{equation}
L=f \frac{N_b^2}{4\pi\sigma^2} R
\end{equation}
where $f$ is the collision frequency, $N_b$ is the number
of particles in a bunch, $R$ is a geometrical factor
taking into account the crossing angle and hourglass effect, 
and $\sigma$ is the transverse size (not cross-section).
$\sigma^2 = \beta^* \varepsilon_N/ \beta\gamma$
where $\beta^*$ is the betatron function (want small), 
$\varepsilon_N$ is the normalized emittance, $\gamma$ is the 
relativistic gamma factor and proportional to energy, $\beta$ is the 
relativistic $\beta$.
LHC design lumi is $10^{34}~\lumiunits$.

useful life of colliding beams is burn rate

luminosity can be increased by making betastar as small as possible
and increasing Nb and epsilone proprotionally



\section{The LHC Accelerator Complex}
list the detectors and the different machines
dipoles
van-der-meer scans
start with hydrogen
RF cavities
design goal of luminosity. luminsoity decays per run


show diagram
\section{Data Collection}

luminosity. uncertainty?
at what luminosity did it actually run?
include some performance and collection graphs
