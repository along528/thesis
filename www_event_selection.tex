\section{Event selection}
\label{sec:eventSelection}

\subsection{Event Pre-selection}
\label{sec:preselection}

The events undergo a first set of event pre-selection cuts to be considered in the analysis. In the rest of this document this minimal selection will be referred to as ``event pre-selection''. Events must pass each of the following cuts:

\begin{itemize}
	\item Good run list: As stated in Section~\ref{sec:subsection_data}, the data must pass a selection ensuring that the detector and LHC conditions were good enough to introduce no selection bias.
	\item Event cleaning: The events where the LAr or Tile calorimeters have had functional problems during the data taking are explicitly vetoed. This is done using the following logic:
	\begin{itemize}
		\item Noise bursts in LAr calorimeters: reject if \url{larError==2}
		\item Corrupted events in the tile calorimeter: reject if \url{tileError==2}
		\item Events where tile drawer were off: reject if \url{TTileTripReader::checkEvent()==1}
		% \item Events where tile drawer were off: reject if \url{TTileTripReader::checkEvent(unsigned\ int\ run,\ unsigned\ int\ lbn, unsigned int event)}		
	\end{itemize}
		
	\item Primary vertex: The event must contain a primary vertex with 3 tracks.
	\item \met{} cleaning: If an event contains jets close to a badly behaving calorimeter region, it is vetoed, in order to avoid any bias on the \met{} measurement.
	\item Trigger: The event must pass one of the lowest un-prescaled single lepton trigger items: EF\_e24vhi\_medium1, EF\_e60\_medium1 or EF\_mu24i\_tight, or EF\_mu36\_tight. Tests were done to add dilepton trigger items with lower \pt\ threshold, but they turned out to increase the signal selection effiency by a minimal amount ($\approx{}5\%$), while this considerbly complicating the analysis. They were therefore not considered further.
	\item 3 leptons selection: The event selected must contain exactly three leptons with $\pt>20~\GeV$.
	\item Trigger matching:	At least one of the leptons must have fired the trigger, and it is required that it's transverse momentum is greater than the threshold fired adding $1~\GeV$ if it does. As stated earlier, the events are corrected for trigger efficiencies measured in data.
\end{itemize}

	

% An event is considered in the analysis if it follows the usual data
% quality requirements used in the ATLAS analysis.  The event must
% belong to a run and a luminosity block, where the detector and LHC
% conditions were found to be good. The events where the LAr or Tile
% calorimeters are misbehaving are explicitly vetoed.  If an event
% contains jets close to a badly behaving calorimeter region, it is
% vetoed, in order to avoid any bias on the \met measurement.
%
% The event selected must contain exactly three leptons with a
% $\pt>15~\GeV$. They must pass one of the lowest prescaled single
% lepton trigger items: EF\_e24vhi\_medium1, EF\_e60\_medium1 or
% EF\_mu24i\_tight, or EF\_mu36\_tight. At least one of the leptons must
% have fired the trigger, and it is required that it has $\pt>25~\GeV$
% if it does.

% The analysis is further categorized based on the number of Same Flavor
% Opposite Signal (SFOS) pairs present in the event.

% \subsection{Optimization}
%
% \clearpage
\subsection{Signal regions}
\label{sec:signal_regions}
%
\subsubsection{Same-Flavor Opposite-Sign Lepton pairs}
\label{sec:sfos}
The signal regions used in this analysis are separated based on the number of 
Same-Flavor Opposite-Sign (SFOS) lepton pairs selected in the event.  That is to say,
the number of lepton pair combinations in the event 
which could feasibly come from the leptonic decay of a $Z$-boson.
This results in three separate signal regions listed below with the lepton charge combinations
which fall in each category:
\begin{itemize}
\item \textbf{0 SFOS}: $e^{\pm}e^{\pm}\mu^{\mp}$, $\mu^{\pm}\mu^{\pm}e^{\mp}$ ($e^{\pm}e^{\pm}\mu^{\pm}$, $\mu^{\pm}\mu^{\pm}e^{\pm}$, $e^{\pm}e^{\pm}e^{\pm}$, $\mu^{\pm}\mu^{\pm}\mu^{\pm}$)
\item \textbf{1 SFOS}: $e^{\pm}e^{\mp}\mu^{\pm}$, $e^{\pm}e^{\mp}\mu^{\mp}$, $\mu^{\pm}\mu^{\mp}e^{\pm}$, $\mu^{\pm}\mu^{\mp}e^{\mp}$
\item \textbf{2 SFOS}: $e^{\pm}e^{\pm}e^{\mp}$, $\mu^{\pm}\mu^{\pm}\mu^{\mp}$
\end{itemize}
Note that in the 2 SFOS region, one lepton is allowed to belong to both pair combinations.
Those combinations listed in parentheses are not allowed for the signal based on charge conservation (neglecting charge mis-identification).  
The amount of the $W^{\pm}W^{\mp}W^{\pm}$ signal
which falls into each category is purely combinatoric.  
From the above list one can thus see that there are twice as many ways for the signal combinations (again neglecting those in parentheses)
to fall in the 1 SFOS regions as there to fall in either the 0 SFOS or 2 SFOS regions. 
Absent possible difference in signal efficiencies based on the leptons in each 
signal region, one should expect branching fractions of 25\%, 50\% and 25\% for the 0, 1, and 2 SFOS signal regions, respectively.

The advantage of splitting the signal region based on this
classification comes when looking at the background, specifically the
electroweak $WZ$ and $ZZ$ backgrounds where SFOS lepton pairs may be
produced from the decay of the $Z$ boson(s). Consider only the case
where the $WZ$ and $ZZ$ decay to either $e$ or $\mu$.  The $WZ$ production
process is thus characterised by 3 leptons with at least 1 SFOS lepton pair
which comes from the $Z$. If all three leptons from the $WZ$ decay have been
reconstructed, then the there is a 50~\% chance the third lepton 
will also be able to form a SFOS pair with one of the leptons from the $Z$ decay.
Thus, the WZ background will split evenly betweeen the 1 and 2 SFOS classification.
Something similar occurs for the ZZ background except that the fourth lepton 
in the decay must be lost (usually due to possesing a low $\pt$).
The large cross-section for theses processes means that
they becomes the dominant backgrounds in the 1 and 2 SFOS regions.  

The 0 SFOS signal region is mostly spared from contamination  by 
these large processes but still
includes both the $WZ$ and $ZZ$ processes as background due to the
non-negligible (albeit small) effect of mis-measurement of the lepton
charge, see section~\ref{sec:chargeMisID}.  The 0 SFOS signal region
is thus unique in having a small background which is almost entirely
reducible and dominated instead by events where a jet is mismeasured
as or overlaps with a lepton, called the fake lepton background, along
with the aforementioned subdominant effect of lepton charge 
mis-identification described in Section~\ref{sec:chargeMisID}.  
From this, one can clearly see that it is
advantageous to split these signal regions so that the dominant
backgrounds in each region may be targeted individually.  Furthermore,
note that while the 1 SFOS region contains more of the signal than the
0 and 2 SFOS regions, it is the 0 SFOS region which is most likely to
have the best sensitivity due to the smaller background contribution.

\subsubsection{Signal Region Selections}

The final signal selection cuts are determined using an optimization
procedure which considers both the signal yield as well as the uncertainty
on the measurement (including statistical and systematic uncertrainties)
of the signal strength as taken from the profile likelihood. 

Many different quantities were considered in the optimization.  All three signal 
regions start from event pre-selection (described above in Section~\ref{sec:preselection}) where 
we have 3 leptons with a $\pt$ of at least 20~GeV 
and where at least one of the leptons is matched to one of the single lepton triggers.
It was considered whether or not to lower the $\pt$ threshold, but this was found to increase
the fake lepton background contribution without gaining much signal. Tighter cuts on the $\pt$
were also considered but this was shown to not be a good discriminant. 

In all regions, 
a veto is applied on events with jets tagged to come from b or b-hadron decays. The highest
efficiency opearting point that is supported by the b-tagging group is used,
which has a b-tagging efficiency of 85~\%.  This also increases the mis-identification efficiency, 
but it remains managable at about 1~\%. Even with some jet mis-identification, the signal has
a very high efficiency of passing this cut of $> 99$~\% while offering some ofthe strongest 
reduction in the fake lepton background. On top of the b-jet veto, there is an additional
cut on the jet multiplicity, regardless of the whether or not the jet is tagged. By only keeping
events with no more than one jet, the signal efficiency is almost $90$~\% while reducing
the background by about 50~\%.  Applying a veto on all jets does a very good job at removing the fake lepton
background, but the signal efficiency is prohibitively small, at about 60~\%. 

A Z-veto is applied in each signal region, 
each which have slightly different specifications and mass windows
as chosen by the optimization. In the 1 and 2 SFOS regions, the Z-veto is performed
by looking at the invariant mass of SFOS pairs. In the 2 SFOS region, both sets of pairs
are considered and the event is vetoed if either fall within the mass window.  In the 
1 SFOS region, there is a larger contribution from $Z\gamma$ processes than in the 2 SFOS
region.  This process mostly shows up in the low shoulder of the Z peak. The optimization
prefers removing this $Z\gamma$ contribution by setting an asymmetric Z-window in the 1 SFOS
region, with the boundaries being 35~GeV below the Z-pole and 20~GeV above. In the 2 SFOS region,
the $Z\gamma$ contribution is not as prominent and the optimization happens to prefer a symmetric
window of $\pm20$~GeV around the Z-pole.  In the 0 SFOS region there are no SFOS pairs by definition,
but there is still a peak in the same-sign electron-electron mass distribution due to charge mis-identification.
The optimization prefers a slightly narrower symmetric window of $\pm15$~GeV around the Z-pole. In all cases
the mass used for the Z-pole is $m_{Z}=91.1876$~GeV as taken from the PDG~\cite{PDG:2014}. 

The signal is characterized by having a large missing $E_{T}$ component.  Therefore, we also 
optimized our the threshold for selecting this quantity separately for each signal region. 
In the 1 and 2 SFOS regions, a missing $E_{T}$ threshold is preferred around $40-60$~GeV.
The missing $E_{T}$ cut also does a good job of cutting out some of the $Z\gamma$ background in the 1 
SFOS region.  As a result, the Z-veto and missing $E_{T}$ cuts are correlated.  The optimization
procedure takes into account this correlation by cutting considering varitions of the Z-veto window
and missing $E_{T}$ threshold together.  As a result, the optimization prefers to keep the missing $E_{T}$ 
cut a little loose in the 1 SFOS region, with a threshold of $E_{T}^{Miss} > 45$~GeV,
while applying the asymmetric Z-veto already discussed as the best 
way to get rid of the $Z\gamma$ contribution.  In the 2 SFOS region the looser Z-veto allows for a tighter
missing $E_{T}$ cut with a threshold of $E_{T}^{Miss} > 55$~GeV. The 0 SFOS region is peculiar in that it is 
dominated by the fake lepton background.  This background turns out to have a similar missing $E_{T}$ distribution
as the signal. As a result, cutting on the missing $E_{T}$ in this region offers little to no discriminating power
between the signal and background so we have chosen not to apply any cut here in order to maximize the signal yield.


The signal is characterized by having three charged leptons and three neutrinos.  The magnitude and direction
of the missing $E_{T}$ may be intepreted as coming from the vector sum of the neutrinos.  By agruments of 
symmetry, one could then compare the azimuthal direction of the missing $E_{T}$ to the azimuthal direction of the vector
sum of the three charged leptons. When doing so, one finds that in the transverse plane, 
the direction of the three charged leptons
tends to be back-to-back with the direction of the three neutrinos (missing $E_{T}$). To some extent, the
backgrounds also show this behavior, but it is less pronounced than it is for the signal.  As a result, 
there is some discriminating power when cutting on the difference in the two angles: 
$|\Delta\phi(3l,E_{T}^{Miss})| = |\phi(3l)-\phi(E_{T}^{Miss})|$. The behavior of this quantity for signal and
background is similar in all three signal regions so based on the optimization it was chosen to apply the cut
$|\Delta\phi(3l,E_{T}^{Miss})| > 2.5$ everywhere.  We also considered taking the difference in angle between
the missing $E_{T}$ and individual leptons (e.g. the highest $\pt$ lepton) but this was shown to be not 
nearly as effective.  

Finally, the distribution of same-flavor lepton pairs (regardless of sign) was considered in the 0 SFOS 
region to remove any low-mass contamination from processes like from QCD.  This was shown to offer some
modest discriminating power and a threshold of $m_{SF} > 20$~GeV was chosen only for the 0 SFOS region.


The optimized selection for each signal region is summarized in Table~\ref{tab:signal_selection}.
More details about the optimization procedure and the cuts used 
can be found in appendix~\ref{sec:optimization}.

\begin{table}[ht!]
\centering
\begin{small}
\begin{tabular}{|c||c||c||c|}
\hline
&  0 SFOS  	& 1 SFOS		  & 2 SFOS  \\
\hline 
\hline 
\multirow{2}{*}{Pre-selection} & \multicolumn{3}{c|}{Exactly 3 leptons with $P_{T} > 20$~GeV}\\
                               & \multicolumn{3}{c|}{where at least one is trigger matched.  (See Section~\ref{sec:preselection}) }\\
%\hline
%Lepton $P_{T}$ 	&       \multicolumn{3}{c|}{$P_{T} > 20$~GeV}   	  \\
\hline 
b-tagged Jet Veto	& \multicolumn{3}{c|}{$N_{b-jet} = 0$ (85 \% b-tagging efficiency)} \\
\hline 
Same-Flavor Mass &	$m_{\textrm{SF}} > 20$~GeV	& \multicolumn{2}{c|}{} \\
\hline 
Z-Veto                &  $|m_{ee}-m_Z|$ & $m_{\textrm{SFOS}} < m_{Z}-35\GeV$ & $|m_{\textrm{SFOS}}-m_Z|$ \\
($m_Z = 91.1876\GeV$  &  $>15\GeV$                                         & OR   &  $>20\GeV$\\
                      & 					  & $m_{\textrm{SFOS}}>m_{Z}+20\GeV$	   &  \\
%Z-Veto                &  \multirow{3}{*}{$|m_{ee}-m_Z| > 15$~GeV} & $m_{\textrm{SFOS}} < m_{Z}-35\GeV$ & \multirow{3}{*}{$|m_{\textrm{SFOS}}-m_Z| > 20$~GeV} \\
%($m_Z = 91.1876\GeV$  &                                           & OR                                     &  \\
%                      & 					  & $m_{\textrm{SFOS}}>m_{Z}+20\GeV$	   &  \\
\hline 
Missing $E_{T}$		& 		& $E_{T}^{Miss} > 45$~GeV & $E_{T}^{Miss} > 55$~GeV \\
\hline 
Lepton-Missing $E_{T}$ Angle 	& 	\multicolumn{3}{c|}{$|\phi(3l)-\phi(E_{T}^{Miss})| > 2.5$} \\
\hline 
Inclusive Jet veto	& \multicolumn{3}{c|}{$N_{jet} \leq 1$} \\
\hline 
\end{tabular}

\end{small}
\caption{Optimized signal selection split by number of Same-Flavor Opposite-Sign (SFOS) lepton pairs.}
\label{tab:signal_selection}
\end{table}

\clearpage
\subsubsection{Fiducial Cross-sections}
\label{sec:fiducial_cross_section}


Fiducial regions are defined for each channel which are designed to be close to the reconstruction
level signal selection from Table~\ref{tab:signal_selection}.  
The fiducial selections are determined at truth level using Rivet \cite{Buckley:2010ar}.
%copied from semi-leptonic note since we are using the same code.
Only prompt leptons (those not originating from hadron decays) are used for lepton selections, and these leptons are dressed with prompt photons within a cone with $\Delta R = 0.1$. Generator-level jets are reconstructed by running the anti-kt algorithm with radius parameter $\Delta R = 0.4$ on all final-state particles after the parton showering and hadronization with the exception of prompt leptons, prompt photons, and neutrinos. The MET variable is calculated using all generator-level neutrinos. 
Events are removed if $\tau$ leptons are present from the $W$ decays.  Thus, the fiducial cross-section
does not include the branching fraction to $W\rightarrow\tau\nu$ decay, even though there will be some contamination from this process in the final reconstruction level selection. The final fiducial selection is presented
for each channel in Table~\ref{tab:fiducial_selection}.

%i could put the actual fiducial cross-sections here
%put the cutflows in the backup

\begin{table}[ht!]
\centering
\begin{footnotesize}
\begin{tabular}{|c||c||c||c|}
\hline
&  0 SFOS  	& 1 SFOS		  & 2 SFOS  \\
\hline 
\hline 
Tau Veto & \multicolumn{3}{c|}{$N_{\tau} < 1$} \\
\hline 
Fiducial Leptons & \multicolumn{3}{c|}{Exactly 3 leptons with $p_{T} > 20~\mathrm{GeV}$ and $|\eta|<2.5$} \\
\hline 
Lepton Overlap Removal & \multicolumn{3}{c|}{$\Delta R(\ell \ell) > 0.1$}\\
\hline 
Same-Flavor Mass &	$m_{\textrm{SF}} > 20$~GeV	& \multicolumn{2}{c|}{} \\
\hline 
Z-Veto                &  $|m_{ee}-m_Z|$ & $m_{\textrm{SFOS}} < m_{Z}-35\GeV$ & $|m_{\textrm{SFOS}}-m_Z|$ \\
($m_Z = 91.1876\GeV$)  &  $>15\GeV$                                         & OR   &  $>20\GeV$\\
                      & 					  & $m_{\textrm{SFOS}}>m_{Z}+20\GeV$	   &  \\
\hline 
Missing $E_{T}$		& 		& $\MET > 45$~GeV & $\MET > 55$~GeV \\
\hline 
Lepton-Missing $E_{T}$ Angle 	& 	\multicolumn{3}{c|}{$|\phi(3l)-\phi(E_{T}^{Miss})| > 2.5$} \\
\hline 
Inclusive Jet veto	& \multicolumn{3}{c|}{$\njet \leq 1$ with fiducial jets of $p_{T} > 25~\mathrm{GeV}$ and $|\eta| < 4.5$ } \\
\hline 
\end{tabular}

\end{footnotesize}
\caption{Fiducial regions based on optimized selection.}
\label{tab:fiducial_selection}
\end{table}

%need to add references to the madgraph samples in the sample section
The fiducial cross-sections are evaluated in the VBFNLO samples mentioned in Section~\ref{sec:signal} and compared
to another set of samples generated using MadGraph at NLO in QCD. Unlike the VBFNLO signal samples, which are 
generated only include the leptonic decays of the W boson, the MadGraph samples include all W boson decays, including hadronic decays. The VBFNLO samples, generated at LO, include interference effects between non-resonant production to three Ws and to resonant production to $WH\rightarrow WWW(*)$ while the NLO MadGraph samples do not, generating the two separately. As measured at truth level from the MadGraph samples, resonant production is observed to contribute about 50-60\% to
the fiducial cross-section in each channel, even though it contributes about 90\% of the total cross-section.  A summary of the signal sample phase space and cross-sections is
presented in Table~\ref{tab:signal_sample_summary}.

The derived fiducial cross-sections using the two generators are shown in Table~\ref{tab:fiducial_cross_sections}.  The cross-sections are observed
to be in good agreement between the two generators. The MadGraph NLO
fiducial cross-sections are used in the final measurement.
% it would probably be a good idea to include the fiducial cutflows as well.



\begin{table}[ht!]
\centering
\begin{footnotesize}
\begin{tabular}{|cc||c|c|c|}
\hline
\multicolumn{2}{|c||} {Sample} &  Cross-section [fb] & Lepton $\pt$ & Lepton $\eta$ \\
\hline
\hline
\multirow{2}{*}{VBFNLO LO} & $W^{+}W^{+}W^{-}\rightarrow l\nu l\nu l\nu$ & 4.95 & $\pt > 5$~GeV & $|\eta<45|$ \\
                        & $W^{-}W^{+}W^{-}\rightarrow l\nu l\nu l\nu $& 2.65 & $\pt > 5$~GeV & $|\eta<45|$ \\
\hline
\multirow{4}{*}{MadGraph NLO} & $W^{+}W^{-}W^{+}\rightarrow \textrm{Anything}$ &$55.07\pm0.10$ & \textrm{Inclusive} & \textrm{Inclusive} \\
                        & $W^{-}W^{+}W^{-} \rightarrow \textrm{Anything}$& $25.99\pm0.07$ & \textrm{Inclusive} & \textrm{Inclusive} \\
                        & $W^{+}H\rightarrow W^{+}W^{+}W^{-}(*)\rightarrow\textrm{Anything}$ & $91.765\pm0.018$ 
			& \textrm{Inclusive} & \textrm{Inclusive} \\
                        & $W^{-}H\rightarrow W^{-}W^{+}W^{-}(*) \rightarrow \textrm{Anything}$& $50.7440\pm0.0094$ & \textrm{Inclusive} & \textrm{Inclusive} \\

\hline
\end{tabular}

\end{footnotesize}
\caption{Details of signal samples used to study signal fiducial cross-sections.}
\label{tab:signal_sample_summary}
\end{table}





\begin{table}[ht!]
\centering
\begin{footnotesize}
\begin{tabular}{|c||c|c|}
\hline
        & \multicolumn{2}{|c|}{Fiducial Cross-section [ab]} \\
Channel & MadGraph & VBFNLO \\
\hline\hline
0 SFOS &  $114.7 \pm 4.3$     & $126.9 \pm 1.0$      \\
1 SFOS &  $126.6 \pm 4.3$     & $126.1 \pm 1.0$       \\
2 SFOS &  $50.2 \pm 2.7$     & $50.62 \pm .66$       \\
\hline
\end{tabular}

\end{footnotesize}
\caption{Fiducial cross-sections derived in each signal region for the two generators. Production modes are summed together to get one fiducial cross-section per channel per generator. The cross-sections are seen to be in good agreement
between the two generators.}
\label{tab:fiducial_cross_sections}
\end{table}
