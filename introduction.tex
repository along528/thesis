There was a moment when the world thought they knew everything
about the matter of the universe. That was in February of 1932, when Chadwick discovered the
neutron \cite{Chadwick:1932ma}. By that point the proton had already been 
well established by Rutherford a decade prior as 
making up the nucleus of the Hydrogen
atom \cite{Rutherford:1911zz}. 
And the electron had been known to be positioned in the atom
since Thompson established their particle nature a couple of decades before
that \cite{thompson:electron}. 
The neutron helped explain the extra mass in the nucleus of heavier
atoms, like Helium, and that was it. There was the peculiar discovery 
by Compton in 1923 \cite{PhysRev.21.483} 
that light had a particle nature, in the form of a photon,
but there was apparently no need for anything else. 


But then things began to unravel. On August $2^{\textrm{nd}}$, 1932, Anderson
discovered what appeared to a positive electron (now the positron), 
and in March of 1933 published 
the photograph to prove it \cite{PhysRev.43.491}\footnote{If only it were still so easy.}. 
Well, that could be explained nicely by Dirac's relativistic equations
for the electron. But then in 1936, Anderson and Neddermeyer discovered
what is now know to be the muon \cite{PhysRev.51.884}, 
leading one famous Nobel Laureate
to proclaim ``Who ordered that?'' And then there were more.
In 1947, the charged pion was shown to be distinct 
from the muon \cite{Lattes:1947my}.
Another brief moment where they though they were done and then the unstable
charged and neutral Kaons were discovered in the 
same year \cite{Rochester:1947mi} implying
the existence of some new class of particles (now associated with the discovery 
of the strange quark). By this point things were getting messy with new particles
being discovered regularly and efforts being taken to classify them all.
%In 1956, the electron neutrino was discovered \cite{Cowan:1992xc}, but was expected from Fermi's theory of the missing momentum in the weak interactions.  So the discovery of the tau lepton in 1976 \cite{Perl:1977su} didn't cause as much of a fuss at this point.


Then came the \emph{November Revolution} with the discovery 
of the $J/\psi$ in November of 1974 \cite{PhysRevLett.33.1404,PhysRevLett.33.1453}.
Its existence couldn't be explained by the current class 
of models, and required an entirely
new model. This lead 
to the introduction of the 
quark model of the hadrons and the proposal of another quark, the charm.  
This predicted an entirely new slew of particles, but at least 
they could be explained
now instead by a short list of constituent particles, now known as the 
quarks: the up, down,
strange, and charm, as well as the bottom and top discovered later in 
1977 \cite{Herb:1977ek}
and 1995 \cite{PhysRevLett.74.2632,PhysRevLett.74.2626}, respectively.


A whirlwind of theoretical progress occurred in the mid-$20^{\textrm{Th}}$ century
to explain all of the new particles popping up from the experiments. 
But it was the unification of the weak and electromagnetic interactions
by Glashow, Weinberg, and Salam \cite{glashow:1961tr,Salam:1968rm,weinberg:1967tq}
as well as the proposal for giving
mass to the weak gauge bosons by Englert, Brout, Higgs, Guralnik, Hagen,
and Kibble 
\cite{PhysRevLett.13.321,PhysRevLett.13.508,PhysRevLett.13.585}
in the 1960s that has informed much of the 
experimental effort in particle
physics ever since. The $W$ and $Z$ bosons predicted by the theory were 
discovered in 
1983 at CERN \cite{ARNISON1983103,ua1zobs}. 
And most recently, the Higgs boson was 
discovered in 2012 \cite{Aad20121,Chatrchyan:2012xdj}
nearly 50 years after it was first proposed. Together with the 
theory of Quantum Chromodynamics, this ostensibly completes the Standard 
Model (SM) of particle physics. 
So are we done? Just like they thought in 1932?


Perhaps not. There are experimental hints of entirely new realms of the universe, 
so mysterious we refer to them as dark matter and dark energy (cite). 
If we could only find a new particle, this generation's version of the positron
or the $J/\psi$, we could potentially open the door to this new realm.
Perhaps the key is lurking somewhere in the observations of the SM and we 
just haven't been 
able to find it yet.  Much of the structure
of the SM, like why there are three generations of 
leptons and quarks, are not understood.
And there is still uncharted territory within the SM itself.

We seek to explore this uncharted territory by looking for a 
process which is predicted
by the SM but it is too rare to have yet been observed.  Namely, the 
production of 
three $W$ bosons simultaneously from proton-proton collisions. The massive 
nature of the 
$W$ ($m_W = 80.385 \pm 0.015\GeV$~\cite{PDG:2014}) requires a tremendous amount 
of energy 
to produce it. This is why it was not observed until 1983, when a 
collider with sufficient
energy ($540 \GeV$) could be built. So naturally, producing more than one requires
even more energy. Production of two $W$ bosons has been measured most recently 
by ATLAS \cite{Aad:2016wpd} and CMS \cite{Chatrchyan:2013oev}, but 
the production of three $W$ bosons has remained out of reach. 


The high energy and rates of collisions of the Large Hadron Collider (LHC) at CERN
provide an opportunity to look for this process for the first time. 
Indeed, the data collected in the LHC in 2012 should have produced around 150 of collisions
of this type on average, which could potentially be measured by any of the detectors
at the LHC, though we use the ATLAS detector, which is particularly well suited 
for this type of measurement. There are, however, many other processes that are produced
far more copiously and so can easily swamp any sensitivity to this process. 
Thus, we must carefully sift through the data, trying to separate our signal
from the background.  Once we think we've done our best, we assess 
whether or not the signal is present by measuring the cross-section of the process. 
We also use these results to assess whether or not new physics processes have been 
observed, and if not, whether we can rule out any class of new physics models. 
In the words of one Professor John Butler, ``What could go wrong?''


As this thesis is attempting to measure a prediction of the SM, it starts
with a discussion of the details of the SM itself and a description of the 
$WWW$ process in Chapter \ref{sec:theory}. It introduces both the theory 
of Quantum Chromodynamics (QCD) and Parton Distribution Functions (PDF), both
of which are important at a proton-proton collider like the LHC.  
This is followed by 
a description of the Electroweak (EW) theory, with a particular emphasis
on how it leads to the predictions of the $WWW$ process. The experimental
apparatuses of the LHC and ATLAS are described in Chapters \ref{sec:lhc} and
\ref{sec:atlas}, respectively. Chapter \ref{sec:lhc}
describes the general principles of particle physics performed at a hadron collider
as well as specifics of the LHC and the dataset that was used for this thesis. 
Chapter \ref{sec:atlas} describes the different components of the ATLAS detector
and how they are used to identify and measure the properties of the products
of the collisions of the LHC.  Once these tools have been described, 
we get to the main thrust of the thesis, which is how we use them to search for the 
$WWW$ process.  Chapter \ref{sec:www} goes into precise detail about how
we seek to find the $WWW$ process in one particular decay channel where 
each $W$ decays into electrons or muons plus a neutrino. It begins
by introducing the 
dataset as well as  the simulations that are used to predict the 
signal and background.
It then talks about how the collisions are selected to obtain a good 
collection of signal.
One of the most challenging aspects of an analysis of this type is 
being sure
that one understands all of the backgrounds to the process. Thus, it is 
next devoted
to how the backgrounds are estimated. This is followed by a presentation of 
the observed
data after final selection and how this compares to the signal plus 
background estimates. 
Using this information, the results are then interpreted to extract a cross-section
measurement and uncertainty on the signal process and to extract information about
new physics as predicted by an effective field theory (EFT) approach using so-called 
anomalous Quartic Gauge Couplings (aQGC) in just this decay channel. In Chapter
\ref{sec:combination}, this is 
followed by a re-interpretation of the sensitivity to the $WWW$ process
after combining with another decay channel where one of the $W$ 
bosons  decays instead
to quarks. This decay channel is not the focus of this thesis, but its 
inclusion can
improve the sensitivity to the $WWW$ process. Thus, the decay channel along with
the results of the search in this individual channel are briefly 
introduced, followed
by a presentation of an updated interpretation of the cross-section 
measurement and 
limits on new physics similar to that which is presented 
in Chapter \ref{sec:www}.
Finally, we conclude in Chapter \ref{sec:conclusion}.

Throughout the thesis we use natural units where $c = \hbar = 1$. 
This takes advantage
of the relativistic relationship between energy, momentum, and mass so that they 
are all in units of the energy, measured in electron volts, or eV. 
The electric charge, $e$, is related
to the fine structure 
constant, $\alpha \approx 1/137$, by $e = \sqrt{4 \pi \alpha}$.


