There was a moment when the world thought they knew everything
about the matter of the universe. That was in February of 1932, when Chadwick discovered the
neutron (cite). By that point the proton had already been 
well established by Rutherford a decade prior as 
making up the nucleus of the Hydrogen
atom (cite). And the electron had been known to be positioned in the atom
since Thompson established their particle nature a couple of decades before
that (cite). The neutron helped explain the extra mass in the nucleus of heavier
atoms, like Helium, and that was it. There was the peculiar discovery 
by Compton in 1923 (cite) that light had a particle nature, in the form of a photon,
but there was apparently no need for anything else. 


But then things began to unravel. On August $2^{\textrm{nd}}$, 1932, Carl Anderson
discovered what appeared to a positive electron (now the positron), and in March of 1933 published 
the photograph to prove it (cite)\footnote{If only it were still so easy.}. 
Well, that could be explained nicely by Dirac's relativistic equations
for the electron (cite). But then in 1936, Anderson and Neddermeyer discovered
what is now know to be the muon (cite), leading one famous Nobel Laureate
to proclaim ``Who ordered that?'' (cite). And then there were more.
In 1947, the charged pion was shown to be distinct from the muon (cite).
Another brief moment where they though they were done and then the unstable
charged and neutral Kaons were discovered in the same year (cite) implying
the existence of some new class of particles (now associated with the discovery 
of the strange quark). By this point things were getting messy with new particles
being discovered regularly and efforts being taken to classify them all.
In 1956, the electron neutrino was discovered (cite), 
but was expected from Fermi's theory of the missing momentum 
in the weak interactions (cite).  And the discovery of the tau lepton
in 1976 (cite) didn't cause too much of a fuss at this point.


On the other hand, the discovery of the $J/\psi$ in November of
1974 (cite) was considered to be such a pivotal moment
that it is referred to as the \emph{November Revoluion}. 
Its existence couldn't be explained by the current class of models, and required an entirely
new model. This lead 
to the introduction of the 
quark model of the hadrons and the proposal of another quark, the charm.  
This predicted an entirely new slew of particles, but at least they could be explained
now instead by a short list of constituent particles, now known as the quarks: the up, down,
strange, and charm, as well as the bottom and top discovered later in 1977 (cite)
and 1995 (cite), respectively.


A whirlwind of theoretical progress occured in the mid-$20^{\textrm{Th}}$ century
to explain all of the new particles popping up from the experiments. 
But it was the unification of the weak and electromagnetic interactions
by Glashown, Weinberg, and Salam (cite) as well as the proposal for giving
mass to the weak gauge bosons by Englert, Brout, Higgs, Guralnik, Hagen,
and Kibble (cite) in the 1960s that has informed much of the experimental effort in particle
physics ever since. The $W$ and $Z$ bosons predicted by the theory were discovered in 
1983 at CERN (cite). And most recently, the Higgs boson was discovered in 2012 (cite),
nearly 50 years after it was first proposed. Together with the theory of Quantum Chromodynamics
proposed by ... (cite), this ostensibly completes the Standard Model (SM) of particle physics. 
So are we done? Just like they thought in 1932?


Perhaps not. There are experimentl hints of entirely new realms of the universe, 
so mysterious we refer to them as dark matter and dark energy (cite). 
If we could only find a new particle, this generation's version of the positron
or the $J/\psi$, we could potentially open the door to this new realm.
Perhaps the key is lurking somewhere in the observations of the SM and we just haven't been 
able to find it yet.  Much of the structure
of the SM, like why there are three generations of leptons and quarks, are not understood.
And there is still uncharted territory within the SM itself.

We seek to explore this uncharted territory by looking for a process which is predicted
by the SM but it is too rare to have yet been observed. It is the production of 
three $W$ bosons simultaneously from proton-proton collisions. The massive nature of the 
$W$ ( $m_W = 80.385 \pm 0.015\GeV$~\cite{PDG:2014} ), requires a tremendous amount of energy 
to produce it. This is why it was not observed until 1983, when a collider with sufficient
energy ($540 \GeV$) could be built. So naturally, producing more than one requires
even more energy. Production of two $W$ bosons was observed in ... (cite), but 
the production of three $W$ bosons has remained out of reach. 
The high energy and rates of collisions of the Large Hadron Collider (LHC) at CERN
provide an opportunity to look for this process for the first time. 







neutrinos

getting messy
Eventually, so many new particles had been discovered things 

Before they knew it.


We now have.


give big picture plus some context

quick trip through the whole project

1. intro to intro with short version of whole intro
2. full context. how much depends on what follows. i.e. i don't need to describe the sm in detail since it is done later
3. restatement of problem with full context
4. restatement of response
5. roadmap


general to specific to general

make it unexpected

context: discovery of electron (?), then we discovered it is far more complicated, this led to theoretical revolution in 60s and discoveries. ever since we have been trying to fill in the last pieces. the higgs was discovered. now what?
some parts of the sm still haven't been studied.
i will study it

%there's a fine line between clever and stupid

%Within the past six years much has changed in the world of particle physics. The mechanism for electroweak symmetry breaking was confirmed in 2012 with the discovery of the Higgs boson. At this very moment, tantalizing hints of new discoveries are appearing. Will this reveal something new and usher in a new era in the world of particle physics? We will know soon enough.  Regardless, we must continue to look if we are to realize our own humanity.


%This section should give a general history of the science leading up to modern particle and collider physics

%It should set the stage for the thesis and step through a brief description of what will go 
%into each chapter.  

%I should probably write this last

%Doesn't need to be more than a couple of pages




