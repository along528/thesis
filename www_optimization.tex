\subsection{Optimization}
\label{sec:optimization}

A more stringent selection must be applied on top of the pre-selection
in \sec\ref{sec:preselection_yield} in order to obtain any
sensitivity to the signal.  The best selection, however, is
not known \emph{a priori}. We try to find the best possible selection
by starting from a list of kinematic quantities, chosen based on heuristic
arguments. These kinematic quantities, along with the signal
plus background model, are passed into an optimization 
framework that systematically seeks to simultaneously 
maximize the predicted signal and the precision on the final measurement.

The optimization framework considers independent permutations of the different
kinematic quantities, along with variations of the selection thresholds
on these cuts, to form combinations of selection cuts which could become a final
selection. For each combination, or operating point,
that is considered, the signal plus background
model is evaluated to determine the expected yields and systematic
uncertainties given that selection.  The LHC data is not used in the optimization.  
The prediction is then plugged into the statistical framework described in
\sec\ref{sec:measurement} to extract a value on the expected precision
of the measurement. For each operating point,
the value on the precision is plotted against the expected signal yield.
With some discretion, we then choose the operating point 
that maximized the signal yield and gives the smallest absolute 
precision on the measurement. 


The selection thresholds considered are...

We split by signal region...

The different operating points look like...


Some of the intuition for this can be understood by looking at individual
distributions...

this results in the selection of table blah...
